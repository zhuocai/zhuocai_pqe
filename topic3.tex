\section{Future Work: Alternative Solutions to Multi-Threshold Threshold Cryptosystem} \label{sec:multi-threshold}

Traditional secret-sharing based threshold cryptosystem usually has a fixed threshold $t$. Researchers proposed to include dummy parties for flexible threshold. However, these schemes~\cite{DBLP:conf/crypto/DelerableeP08} require a threshold signature to specify the set of signers $S$ that generate the signature. In the verification step, the verifiers must know the set $S$, so that the verification time is $O(n)$. This is not significantly different from the na\"ive solution 

A direction that is worth exploring is to find a new light-weight cryptographic primitive for dynamic-threshold threshold signatures that allows (1) generating a signature of sublinear/constant size, (2) verifying any signature in sublinear/constant time using a verification key of sublinear/constant size. By light-weight, it cannot use complicated primitives like SNARKs. I conject that there is a efficient solution. The solution should be highly related to secret sharing, since the efficiency of Schnorr/BLS threshold signature mostly comes from the nice properties of secret sharing. However, discrete log based cryptography might not be compatible. In discrete log cryptography, there is a clear distinction between an exact value and a sample from uniform distribution. For example, a signature is either valid or invalid depending on an equality check $e(H(m), y)=e(\sigma, g)$. If we can define a continuous distance measure between a real signature $\sigma_{GT}$ and other elements in the signature space that there exist weak signatures $\sigma_{weak}$ relatively close to $\sigma_{GT}$ compared to the distance between a random element and $\sigma_{GT}$. Lattice-based cryptography might be inspiring because they work with errors. 

\paragraph{Lattice-based threshold signatures} Lattice cryptography relies on completely different hard problems from discrete log. For the purpose of encryption and digital signatures, lattice based solutions are not as efficient as discrete-log based solutions. However, lattice cryptography attracts research attention mainly because (1) the only fully homomorphic encryption scheme so far is based on lattice problems, and (2) lattice problems are conjectured to be quantum-safe (though challenged by a recent work \cite{cryptoeprint:2024/555}), while there are efficient quantum algorithms to solve discrete-log problems. Since Boneh etc. proposed the first lattice-based threshold signature scheme~\cite{DBLP:conf/crypto/BonehGGJKRS18}, there have been many works that improve the practical efficiency of lattice-based threshold signatures. However, to my best knowledge, all existings works have not provided an efficient solution for the dynamic threshold setting. 