
\section{Introduction}

\paragraph{Motivation of classical threshold signatures} The early works of threshold signatures are tightly related to the general concept of threshold cryptosystems. In companies, or committees, a group of members might jointly represent the authority, instead of granting one person absolute power. Therefore, threshold signatures are introduced where all $n$ members jointly maintain the secret key and only a subset of at least $t$ can collaborate to generate a valid signature on behalf of the group. For example, in a committee of 3 members, any member cannot unilaterally make a decision, but any combination of 2 members are allowed to make decisions as long as they collaborate. Threshold cryptosystems are also important in fault-tolerant distributed systems where any node might crash at some time. 

\paragraph{Classical threshold signatures} The most popular threshold signatures are based on Schnorr or BLS digital signature schemes and verifiable secret sharing~\cite{DBLP:journals/joc/BonehLS04,DBLP:conf/eurocrypt/Pedersen91a,DBLP:conf/crypto/CritesKM23}. The goal of threshold signature scheme is to be efficient. It should be much more efficient than the na\"ive solution of concatenating ordinary signatures, which create signatures of $\Theta(n)$ size and require $\Theta(n)$ time for verifiers. Threshold Schnorr signatures require signers to interact with each other for at least two rounds to generate signatures. In contrast, BLS threshold signature is non-interactive but requires additional cryptography assumption, since the underlying BLS signature requires bilinear pairing. They both achieve constant signature size, verification key size and verification time. 

\paragraph{Limitations} In light of new application scenarios, such as cryptocurrencies, the classical threshold signatures have the following limitations: (1) every signer has the same unit weight, (2) the threshold is fixed, (3) the setup step, i.e. distributed key generation (DKG), requires at least $O(n^3)$ communication in asynchronous setting in existing protocols, although the signature is efficient afterwards. SNARK (\textit{Succinct Non-interactive ARguments of Knowledge}), a cryptography primitive that has been close to practical in recent years, allows an alternative paradigm of threshold signatures in weighted and multi-threshold setting. However, SNARK is complicated and usually incurs large concrete cost even though the asymptotic complexity is comparable with classical threshold signatures. 

This survey presents the basic components of threshold signatures in section \ref{sec:basic}, including digital signature schemes, verifiable secret sharing and SNARKs. Section \ref{sec:threshold} presents classical threshold signatures based on RSA, Schnorr and BLS signatures. It also discusses related exotic signature schemes including multi-signature and aggregate signatures. Section \ref{sec:dkg} discusses DKG protocols, which is not only the necessary setup step of threshold signature scheme but also becomes the efficiency bottleneck in dynamic systems. More specifically, we introduce the classical Pedersen's DKG and briefly discuss a recent work on DKG in the challenging asynchronous setting. Next in section \ref{sec:weighted} we present a recent work that designs succinct threshold signatures in weighted and multi-threshold setting, using a specially designed SNARK. In sections \ref{sec:future-aggregate}-\ref{sec:multi-threshold}, we introduce a few possible future directions. 

