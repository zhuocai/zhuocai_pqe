\subsection{Cryptographic Assumptions}

\subsection{Pairing}

\subsection{Digital Signatures}

\begin{definition}{(Digital Signature Scheme)} A digital signature scheme, {\sffamily SGN= (Setup, Gen, Sig, Ver)}, consists of four algorithms defined as follows: 
\end{definition}
\begin{itemize}
    \item {\sffamily Setup$(1^\lambda) \to $ par} takes an input the security parameter $1^\lambda$ and outputs global public parameters {\sffamily par}, where {\sffamily par} defines public key infrastructure and domains of messages and signatures, and all related algorithms implicitly take {\sffamily par} as input.  
    \item {\sffamily Gen(par) $\to $ (pk, sk)} takes as input global parameters {\sffamily par}, and outputs a pair of public/secret keys {\sffamily (pk, sk)}. 
    \item {\sffamily Sig(sk, m) $\to\sigma$} takes as input secret key {\sffamily sk} and a message {\sffamily m}, and outputs a signature $\sigma$. 
    \item {\sffamily Ver(pk, m, $\sigma$) $\to b$} is deterministic, takes as input a public key {\sffamily pk}, a message {\sffamily m}, and a signature $\sigma$,  and outputs a bit $b\in\{0,1\}$ indicating whether the signature is valid or not. 
\end{itemize}

\subsubsection{}

\subsection{Threshold Signatures}

\subsection{Multi Signatures}

\subsection{Aggregators}

\subsection{Weighted Threshold Signatures}

\subsection{Adaptive Security}