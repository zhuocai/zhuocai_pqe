\subsection{Cryptographic Assumptions}

\subsection{Pairing}

\subsection{Digital Signatures}

\begin{definition}{(\textbf{Digital Signature Scheme})} A digital signature scheme, {\sffamily SGN= (Setup, Gen, Sig, Ver)}, consists of four algorithms defined as follows: 
\end{definition}
\begin{itemize}
    \item $\mathsf{Setup(1^\lambda) \to par}$ takes an input the security parameter $1^\lambda$ and outputs global public parameters $\mathsf{par}$, where $\mathsf{par}$ defines public key infrastructure and domains of messages and signatures, and all related algorithms implicitly take $\mathsf{par}$ as input.  
    \item $\mathsf{Gen(par)} \to (\mathsf{pk}, \mathsf{sk})$ takes as input global parameters $\mathsf{par}$, and outputs a pair of public/secret keys $(\mathsf{pk}, \mathsf{sk})$. 
    \item $\mathsf{Sig(sk, m)} \to \sigma$ takes as input secret key $\mathsf{sk}$ and a message $\mathsf{m}$, and outputs a signature $\sigma$. 
    \item $\mathsf{Ver}(\mathsf{pk}, \mathsf{m}, \sigma) \to b $ is deterministic, takes as input a public key $\mathsf{pk}$, a message $\mathsf{m}$, and a signature $\sigma$,  and outputs a bit $b\in\{0,1\}$ indicating whether the signature is valid or not. 
\end{itemize}

A digital signature scheme should satisfy the following requirements:
\begin{itemize}
    \item Correctness: $Pr[(\mathsf{pk}, \mathsf{sk})\leftarrow \mathsf{Gen(par)}, \mathsf{Ver(pk, m, Sig(sk, m))}]=1$. 
    \item Security: without the secret key $\mathsf{sk}$, even if an adversary queries $\mathsf{Sig(sk, \cdot)}$ on a set of messages, it cannot forge signatures for other messages except with negligible probability. Formally, for all probabilistic polynomial time algorithm $\mathcal{A}$, $Pr[(\mathsf{pk}, \mathsf{sk})\leftarrow \mathsf{Gen(par)}, (\mathsf{x},\mathsf{t})\leftarrow \mathcal{A}^{\mathsf{Sig(sk, \cdot)}}(\mathsf{pk}, 1^\lambda), \mathsf{x}\notin Q, \mathsf{Ver(pk, x, t)}=1] = negl(\lambda)$, where $Q$ is the set of messages that $\mathcal{A}$ queries through the oracle $\mathsf{Sig(sk, \cdot)}$. 
    
\end{itemize}

\subsubsection{}

\subsection{Threshold Signatures}

\subsection{Multi Signatures}

\subsection{Aggregators}

\subsection{Weighted Threshold Signatures}

\subsection{Adaptive Security}