\documentclass[11pt]{article}
%
% T1 fonts will be used to generate the final print and online PDFs,
% so please use T1 fonts in your manuscript whenever possible.
% Other font encondings may result in incorrect characters.
%
\usepackage{geometry}
 \geometry{
 a4paper,
 left=25mm,
 right=25mm,
 top=34mm,
 bottom=34mm}
\usepackage{tocloft}
\usepackage{graphicx}
\usepackage{amsmath}
\usepackage{amssymb}
\usepackage{xcolor}
\usepackage{mathtools}
\usepackage{array}

% Used for displaying a sample figure. If possible, figure files should
% be included in EPS format.
%
% If you use the hyperref package, please uncomment the following two lines
% to display URLs in blue roman font according to Springer's eBook style:
%\usepackage{color}
%\renewcommand\UrlFont{\color{blue}\rmfamily}
%\urlstyle{rm}
%

\newtheorem{theorem}{Theorem}[section]
\newtheorem{lemma}[theorem]{Lemma}
\newtheorem{proposition}[theorem]{Proposition}
\newtheorem{corollary}[theorem]{Corollary}

\newenvironment{proof}[1][Proof]{\begin{trivlist}
\item[\hskip \labelsep {\bfseries #1}]}{\end{trivlist}}
\newenvironment{definition}[1][Definition]{\begin{trivlist}
\item[\hskip \labelsep {\bfseries #1}]}{\end{trivlist}}
\newenvironment{example}[1][Example]{\begin{trivlist}
\item[\hskip \labelsep {\bfseries #1}]}{\end{trivlist}}
\newenvironment{remark}[1][Remark]{\begin{trivlist}
\item[\hskip \labelsep {\bfseries #1}]}{\end{trivlist}}

\newcommand{\qed}{\nobreak \ifvmode \relax \else
      \ifdim\lastskip<1.5em \hskip-\lastskip
      \hskip1.5em plus0em minus0.5em \fi \nobreak
      \vrule height0.75em width0.5em depth0.25em\fi}


\begin{document}
%
\title{Threshold Signatures: Efficient Constructions and Applications in Blockchains}
%
%\titlerunning{Abbreviated paper title}
% If the paper title is too long for the running head, you can set
% an abbreviated paper title here
%
\author{Zhuo Cai\\Department of Computer Science and Engineering\\Hong Kong University of Science and Technology}
%
% \authorrunning{F. Author et al.}
% First names are abbreviated in the running head.
% If there are more than two authors, 'et al.' is used.
%
%
\maketitle              % typeset the header of the contribution
%
\begin{abstract}
Threshold signature schemes are important cryptographic primitives. Due to its importance in distributed systems and real applications, threshold signatures have been studied since late 1980s. In recent years, threshold signatures received research attention again, with application in cryptocurrencies, because classical threshold signatures, such as BLS threshold signatures, are not practical for large-scale applications and dynamic distributed systems. Recent hot topics in threshold signatures include (1) succinct signatures using SNARKs, (2) efficient DKG setup protocols for threshold signatures, (3) applications of multi-signatures in blockchain applications and (4) adaptive security, minimal assumptions, and other theoretical aspects of threshold signatures. This survey presents the basic components of threshold signatures and discusses a few recent works on different directions. Finally, the survey suggests some new directions for future research. 
\end{abstract}
%
%
\newpage
\tableofcontents

\newpage
\section{Introduction}
\addcontentsline{toc}{section}{Introduction}

\paragraph{Topic 1} Classical problems cared about by pure cryptographers. 
\paragraph{Topic 2} SNARKs. 
\paragraph{Topic 3} Blockchain applications.  
\paragraph{Topic 4} More interesting topics in threshold signatures itself. 
For example, sign on a collection of data. 


\subsection{Category of papers}

\begin{itemize}
    \item 2024 preprint: HARTS: High-Threshold, Adaptively Secure, and Robust Threshold Schnorr Signatures. 
    \item 2024 Eurocrypt: Threshold Raccoon: Practical Threshold Signatures from Standard Lattice Assumptions
    \item 2023 AsiaCrypt: Threshold Structure-Preserving Signatures 
    \item 2023 CCS: \cite{DBLP:conf/ccs/DasCXNB023}
    \item 2023 CCS: Poster: Signer Discretion is Advised: On the Insecurity of Vitalik's Threshold Hash-based Signatures. Vitalik uses "Lamport Threshold Signatures". An oversimplified threshold signature scheme attacked by simple methods. 
    \item 2023 Crypto: Practical Schnorr Threshold Signatures Without the Algebraic Group Model. 
    \item 2023 Crypto: Fully Adaptive Schnorr Threshold Signatures.
    \item 2023 PKC: Extendable Threshold Ring Signatures with Enhanced Anonymity. 
    \item 2023 S\&P: Threshold Signatures in the Multiverse. 
    \item 2023 S\&P: Threshold BBS+ Signatures for Distributed Anonymous Credential Issuance. 
    \item 2024 Eurocrypt: Twinkle: Threshold Signatures from DDH with Full Adaptive Security. 
    \item 2022 Crypto: Better than Advertised Security for Non-interactive Threshold Signatures. 
    \item 2022 Crypto: Threshold Signatures with Private Accountability. 
    \item 2002 Crypto: Threshold Ring Signatures and Applications to Ad-hoc Groups. 
    \item 2022 CCS: ROAST: Robust Asynchronous Schnorr Threshold Signatures. 
    \item 2020 CCS: Asynchronous Distributed Key Generation for Computationally-Secure Randomness, Consensus, and Threshold Signatures. 
    \item 2001 Eurocrypt: Practical Threshold RSA Signatures without a Trusted Dealer
    \item 2000 Eurocrypt: Practical Threshold Signatures. 
    \item 1996 Eurocrypt: Robust Threshold DSS Signatures. 
\end{itemize}



\newpage
\section{Cryptography Preliminaries and Digital Signatures} \label{sec:basic}

\subsection{Digital Signatures}

\begin{definition}{(Digital Signature Scheme)} A digital signature scheme, {\sffamily SGN= (Setup, Gen, Sig, Ver)}, consists of four algorithms defined as follows: 
\end{definition}
\begin{itemize}
    \item $\mathsf{Setup(1^\lambda) \to par}$ takes an input the security parameter $1^\lambda$ and outputs global public parameters $\mathsf{par}$, where $\mathsf{par}$ defines public key infrastructure and domains of messages and signatures, and all related algorithms implicitly take $\mathsf{par}$ as input.  
    \item $\mathsf{Gen(par)} \to (\mathsf{pk}, \mathsf{sk})$ takes as input global parameters $\mathsf{par}$, and outputs a pair of public/secret keys $(\mathsf{pk}, \mathsf{sk})$. 
    \item $\mathsf{Sig(sk, m)} \to \sigma$ takes as input secret key $\mathsf{sk}$ and a message $\mathsf{m}$, and outputs a signature $\sigma$. 
    \item $\mathsf{Ver}(\mathsf{pk}, \mathsf{m}, \sigma) \to b $ is deterministic, takes as input a public key $\mathsf{pk}$, a message $\mathsf{m}$, and a signature $\sigma$,  and outputs a bit $b\in\{0,1\}$ indicating whether the signature is valid or not. 
\end{itemize}

A digital signature scheme should satisfy the following requirements:
\begin{itemize}
    \item Correctness: $Pr[(\mathsf{pk}, \mathsf{sk})\leftarrow \mathsf{Gen(par)}, \mathsf{Ver(pk, m, Sig(sk, m))}]=1$. 
    \item EUF-CMA Security (Existentially Unforgeable under Chosen Message Attack): without the secret key $\mathsf{sk}$, even if an adversary queries $\mathsf{Sig(sk, \cdot)}$ on a set of messages, it cannot forge signatures for other messages except with negligible probability. Formally, for all probabilistic polynomial time algorithm $\mathcal{A}$, $Pr[(\mathsf{pk}, \mathsf{sk})\leftarrow \mathsf{Gen(par)}, (\mathsf{x},\mathsf{t})\leftarrow \mathcal{A}^{\mathsf{Sig(sk, \cdot)}}(\mathsf{pk}, 1^\lambda), \mathsf{x}\notin Q, \mathsf{Ver(pk, x, t)}=1] = negl(\lambda)$, where $Q$ is the set of messages that $\mathcal{A}$ queries through the oracle $\mathsf{Sig(sk, \cdot)}$. 
    
\end{itemize}

\subsubsection{RSA Signature}
RSA cryptosystem (Rivest-Shamir-Adleman) is based on the difficulty to factor a product of two large primes. RSA signature is directly based on RSA public key encryption scheme. 

\paragraph{Key Generation} Choose two large prime numbers $p$ and $q$. Compute $N=pq$ and $\lambda(N) = \mathrm{lcm}(p-1, q-1)$. Choose an integer $e$ such that $1<e < \lambda(N)$ and $\mathrm{gcd}(e, \lambda(N)) = 1$. Then compute $d=e^{-1}\pmod{\lambda(N)}$. The secret key is $\mathsf{sk}=(N, d)$ and the public key is $\mathsf{pk}=(N,e)$. 

\paragraph{Encryption} Given a plaintext message $m$ such that $0\le m < N$, the encryption of $m$ is $c=m^e\pmod{N}$. 

\paragraph{Decryption} With the secret key $(N, d)$, the plaintext message can be recovered from $c$ as $m = c^d\pmod{N}$. 

The important property of RSA algorithm is that $x^{de}=x\pmod{N}$ for any integer $x$ and $e$ is difficult to compute from $d$. 

\paragraph{Signature} In public key encryption, only the owner of secret key $d$ can run the decryption algorithm. In digital signature schemes, a signature can be generated only by the owner of secret key and can be verified by anyone knowing the public key. The public keys and secret keys of RSA have the same structure, RSA encryption and decryption are the same algorithm using different keys. The signing algorithm of RSA signature is exactly the RSA algorithm using the secret key, while the verification algorithm of RSA signature is the RSA algorithm using the public key. 

\paragraph{Signing} Given as input a message $m$ and the secret key $\mathsf{sk}=d$, the signature is $\sigma= H(m)^d \pmod{N}$, where $H$ is a cryptographic hash function. 

\paragraph{Verifying} Given as input a message $m$, a signature $\sigma$ and the public key $\mathsf{pk}=e$, the verification algorithm returns $1$ if $\sigma^e=H(m)\pmod{N}$. 

\subsubsection{ElGamal Signature}
ElGamal signature scheme is based on algebraic properties of modular exponentiation, together with the discrete logarithm problem. The widely used Digital Signature Scheme is a variant of ElGamal signature scheme. 

\paragraph{Public Parameters} (1) Choose a key length $L$ and a $L$-bit prime number $p$. (2) Choose a cryptographic hash function $H$ with output length $L$ bits. (3) Choose a generator $g$ of the multiplicative group of integers modulo $p$, $Z_p^\times$. 

\paragraph{Key Generation} Choose an integer $x$ randomly from $\{1,\dots, p-2\}$ as secret key $\mathsf{sk}$. The public key $\mathsf{pk}$ is defined as $y=g^x\pmod{p}$. 

\paragraph{Signing} Given as input a message $m$ and secret key $\mathsf{sk}=x$, the signer (1) chooses an integer $k$ randomly from $\{2,\dots, p-2\}$ that is relatively prime to $p-1$, (2) computes $r=g^k\pmod{p}$, (3) computes $s=(H(m)-xr)k^{-1}\pmod{(p-1)}$. The signature is $\sigma=(r,s)$. 

\paragraph{Verifying} Given as input a message $m$, a signature $\sigma=(r,s)$ and public key $\mathsf{pk}=y$, the verifier checks that $0<r<p$, $0<s<p-1$ and $g^{H(m)}=y^r r^s\pmod{p}$. 


\subsubsection{Schnorr Signature}

\paragraph{Parameters} All users of the signature scheme agree on a group $\mathbb{G}$ of prime order $q$, with generator $g$. All users also agree on a cryptographic hash function $H:\{0,1\}^*\to \mathbb{Z}_q$. 

\paragraph{Schnorr Group} Usually Schnorr signatures use a particular class of groups called Schnorr groups. A \textbf{Schnorr group} is a large prime-order subgroup of $\mathbb{Z}_p^{\times}$, the multiplicative group of integers modulo $p$ for some prime $p$. To generate such a group, generate two prime numbers $p, q (p>q)$, such that $p=qr+1$ for an integer $r$. Then choose any $h$ in the range $1<h<p$ such that $h^r\not\equiv 1$. $g=h^r\bmod p$ is a generator of a subgroup of $\mathbb{Z}_p^\times$ of order $q$. 

\paragraph{Digital Signature Algorithm (DSA)} DSA is a variant of ElGamal signature scheme using Schnorr groups instead of the multiplicative groups $\mathbb{Z}_p^\times$. 

\paragraph{Key Generation} Choose a private signing key $\mathsf{sk}=x\in Z_q^\times$. The corresponding public key is $\mathsf{pk}=y=g^x \in \mathbb{G}$. 

\paragraph{Signing} To sign a message $m$ using the secret key $x$, choose a random $k$ from $Z_q^\times$, let $r=g^k$, compute $e=H(r||m)$, where $||$ denotes concatenation of bit strings. Let $s=k-xe$, the signature is the pair $(s,e)$. Note that $s, e\in Z_q^\times$. 

\paragraph{Verifying} To verify that $(s,e)$ is the signature of $m$ using the public key $y$, the verifier computes $r_v = g^s y^e$ and $e_v = H(r_v||m)$. If $e_v=e$, then accept the signature. 


\subsubsection{BLS Signature}
BLS digital signature was proposed by Boneh, Lynn and Shacham \cite{DBLP:journals/joc/BonehLS04}, using a bilinear pairing for verification. It is provably secure (existentially unforgeable under adaptive chosen-message attacks) in the random oracle model assuming the intractability of the computational Diffie-Hellman problem in a gap Diffie-Hellman group. 

\paragraph{Diffie-Hellman Problems and Gap-Diffie-Hellman groups} Let $\mathbb{G}$ be a multiplicative group of the prime order $p$. We consider the following two problems in $\mathbb{G}$:
\begin{definition} {(Computational Diffie-Hellman (CDH) problem)} 
    Given $(g, u, v)$, three random elements of $\mathbb{G}$, to compute $h=g^{\log_g u \log_g v}$. 
\end{definition}

\begin{definition} {(Decisional Diffie-Hellman (DDH) problem)}
    Given $(g, u, v, h)$, four elements of $\mathbb{G}$, which with equal probability can be either all random elements of $\mathbb{G}$ or have the property that $\log_g u = \log_v h$, to output $0$ in the former case and $1$ in the latter case. 
\end{definition}

Gap-Diffie-Hellman (GDH) groups are groups where CDH problem is hard but DDH problem is easy. A series of works on the Weil pairing showed the existence of such GDH groups~\cite{DBLP:conf/crypto/BonehF01}. 

\paragraph{Bilinear Pairing} A mapping $e: \mathbb{G}_1\times \mathbb{G}_2\to \mathbb{G}_T$ is a bilinear pairing for groups $(\mathbb{G}_1, \mathbb{G}_2, \mathbb{G}_T)$ of prime order $q$, if $e(g^x, h^y) = e(g, h)^{xy}$ for generators $g$ of $\mathbb{G}_1$ and $h$ of $\mathbb{G}_2$. Bilinear pairings allow efficiently solving the decisional Diffie-Hellman problem even though the computational Diffie-Hellman problem remains intractable. For a CDH problem instance $(g, g^x, g^y, g^z)$, testing $g^z=g^{xy}$ is equivalent to checking $e(g^x, g^y) = e(g, g^z)$, for a non-degenerate, efficiently computable bilinear mapping from $\mathbb{G}\times \mathbf{G}$ to a target group $\mathbb{G}_T$. 

\paragraph{Key Generation} Choose a private signing key $\mathsf{sk}=x\in Z_q^\times$. The corresponding public key is $\mathsf{pk}=y=g^x \in \mathbb{G}$. 

\paragraph{Signing} The signature for message $m$ is $\sigma=H(m)^x$ using the secret key $x$.  
\paragraph{Verification} To verify that $\sigma$ is the signature of $m$ using the public key $y$, the verifier checks that $e(\sigma, g) = e(H(m), y)$. 

\subsubsection{EdDSA}
In most practical digital signature schemes and more generally public key cryptosystems that rely on the hardness of discrete logarithm, we always work with a multiplicative group $\mathbb{G}$ of prime order $q$ with a generator $g\in\mathbb{G}$. In the presentation of ElGamal signature and Schnorr signature, we use a subset of integers modular a prime number $p$, i.e., we work with $\mathbb{G}\subset \mathbb{Z}_p^\times$. Modern discrete-log based cryptosystems prefer to use elliptic curves as the group, where the discrete log problem has considerably more bits of security. A popular digital signature scheme is Edwards-curve Digital Signature Algorithm (EdDSA), a variant of Schnorr signature based on twisted Edwards curves. The construction of elliptic curves is not covered in this survey. We abstract them as multiplicative groups with generators, where the discrete logarithm problem is hard. 


\subsection{Secret Sharing}
\subsubsection{Shamir Secret Sharing and Lagrange Interpolation}
How to share a secret $s$ among $n$ players such that any subset of at least $t$ players can collaborate to recover $s$ but no subset of fewer than $t$ players can recover $s$ or gain any information about $s$? Each player $i$ should receive a piece of data that is called a share $s_i$ of the secret $s$. The person that owns $s$ and shares it is called the \textbf{dealer}. Shamir's secret sharing is an elegant solution for this problem, that uses the property of univariate polynomials of bounded degrees. 

\paragraph{Construction of Shamir's Secret Sharing} In order to share a secret $s$, the dealer chooses a random polynomial of degree at most $t-1$ over a field $\mathbb{F}$, $f(X)=\sum_{i=0}^{t-1} a_i X^i$, by setting $a_0=s\in \mathbb{F}$ and choosing all other coefficients $a_i(1\le i\le t-1)$ independently at random from $\mathbb{F}$. The secret share for player $i$ is $s_i = f(i)$. The secret is $s=f(0)$. 

\paragraph{Lagrange Interpolation} Suppose a group of $t$ players want to jointly recover the secret $s$ using their shares $\{x_{j_1}, x_{j_2},\dots, x_{j_t}\}$. Let $S = \{j_1, j_2, \dots, j_t\}$. They can interpolate a polynomial of degree at most $t-1$ that passes the points $\{(j_1, x_{j_1}), (j_2, x_{j_2}), \dots (j_t, x_{j_t})\}$ as
\begin{equation*} 
    g(x) = \sum_{k=1}^{t}\frac{\prod_{l \in S\setminus \{j_k\} } (x - l)}{\prod_{l \in S\setminus \{j_k\} } (j_k - l)}\cdot x_{j_k}   
\end{equation*}

\par Then $g$ is exactly the same polynomial as $f$, because they both have degree at most $t-1$ and they evaluate the same at $t$ different locations. Hence, the secret $s$ is computed as $g(0)$. Denote by $L_{S, j_k}(x) = \frac{\prod_{l \in S\setminus \{j_k\} } (x - l)}{\prod_{l \in S\setminus \{j_k\} } (j_k - l)}$, called the Lagrange multiplier of $j_k$ for the interpolation set $S$. It is interesting to see that $\lambda_{S, j_k} = L_{S, j_k}(0)$ only depends on the interpolation set $S$ and $j_k$, but not on the values of $f$ in these points.  

For any group of at most $t-1$ players, they only know at most $t-1$ evaluations of the polynomial $f$, so $s$ can be an arbitrary value in the allowed range. Shamir secret sharing is an important building block of many threshold signature schemes. 

\subsubsection{(Publicly) Verifiable Secret Sharing} \label{subsubsec:vss}
If the dealer is required to prove that the shares are distributed correctly and signers are required to prove that they help reconstruct correctly, the variant of secret sharing is called verifiable secret sharing (VSS). If the proofs can be non-interactive and publicly verified by external parties, then it is called publicly verifiable secret sharing (PVSS). In the setting of distributed key generation in threshold signature schemes, as long as the participants of the system believe that the protocols are executed correctly, they can proceed to generate signatures. Therefore, only VSS is required. We introduce the VSS proposed by Pedersen~\cite{DBLP:conf/crypto/Pedersen91}. 

Before presenting Pedersen's VSS, we introduce commitment schemes. Intuitively, commitment scheme allows a player to commits to a number but keep the number itself secret. 

\begin{definition}{(Commitment Schemes)}
    A \textbf{commitment scheme} for input space $\mathcal{X}$ and commitment space $\mathcal{Y}$ consists of two disjoint stages: the \textbf{commit} stage followed by the \textbf{reveal} stage. In the \textbf{commit} stage, the person who wants to make a commitment about a value $x\in\mathcal{X}$, say Alice, can choose a random number $r\in\{0,1\}^\lambda$ and efficiently compute the commitment $c\in\mathcal{Y}$ as the output of a function $\mathsf{commit}$, given $x$ and $r$ as input. Then Alice can send the commitment $c$ to a receiver, say Bob. Later in the \textbf{reveal} stage, Alice can open the commitment by sending $x$ and $r$ to Bob. Bob computes $c'=\mathsf{commit}(x,r)$ and accepts $x$ as the committed value if $c=c'$. 
\end{definition}

A commitment scheme is generally required to satisfy \textbf{hiding} and \textbf{binding} properties with a security parameter $\lambda$. 

\begin{itemize}
    \item \textbf{hiding}: $c$ reveals no information about $x$. Formally, for any $x\in \mathcal{X}$, for any probabilistic polynomial time (PPT) algorithm $\mathcal{A}$, $Pr[c\leftarrow \mathsf{commit}(x, r), \mathcal{A}(c)=x]=negl(\lambda)$. 
    \item \textbf{binding}: it is computationally infeasible to find two different pairs $(x,r)$ and $(x', r')$ such that $\mathsf{commit}(x, r)=\mathsf{x', r'}$ while $x\neq x'$ or $r\neq r'$. Formally, for any PPT algorithm $\mathcal{A}'$, $Pr[(x', r')\leftarrow \mathcal{A}'(x, r), (x',r')\neq (x,r), \mathsf{commit}(x,r)=\mathsf{commit}(x',r')]=negl(\lambda)$. 
\end{itemize}

\paragraph{Pedersen Commitment Scheme} In a group $G$ of prime order $p$, choose element $g\in G$ and $h\in G$ such that nobody knows $\log_g h$. The $\mathsf{commit}$ function is defined as $\mathsf{commit}(x,r) = g^x h^r$, for a random number $r$ chosen by the committer from $\mathbb{Z}_p$ at random. Interested readers can refer to \cite{DBLP:conf/crypto/Pedersen91} for the proof that this commitment scheme satisfies the hiding and binding properties. 

Pedersen's VSS consists of the following rounds:

\begin{enumerate}
    \item The dealer publishes a commmitment to the secret $x: \mathsf{commit}(x, r)$, for a randomly chosen $r\in\mathbb{Z}_p$. 
    \item Like Shamir's secret sharing, the dealer chooses a polynomial $f\in \mathbb{Z}_q[x]$ of degree at most $t-1$ satisfying $f(0)=x$, and computes $x_i=f(i)$ for $i\in\{1,2,\dots, n\}$. Let $f(x) = x_0 + a_1 x + \cdots + a_{t_1} x^{t-1}$. Besides, the dealer also commits to the coefficients $a_i$ for all $i\in\{1,\dots, t-1\}$, by choosing $g_i\in \mathbb{Z}_q$ at random and broadcasting $c_i\mathsf{commit}(a_i, g_i)$. 
    \item Let $g(x)=r + g_1 x + \cdots + g_{t-1}x^{t-1}$, and let $u_i=g(i)$ for $i\in\{1,2,\dots, n\}$. Then the dealer sends $(x_i, u_i)$ secretly to participant $P_i$ for $i\in\{1,2,\dots, n\}$. 
\end{enumerate}

When $P_i$ receives his share $(s_i, t_i)$, he verifies that 
\begin{equation*}
    \mathsf{commit}(x_i, u_i) = \prod_{j=0}^{t-1} c_i^{i^j}
\end{equation*}  

{\color{red} TODO write some proof here}




\subsection{Adaptive Security}


\subsection{SNARKs}
While secret sharing, a relatively simple cryptographic primitive, is sufficient for succinct threshold signatures (see constructions in section \ref{sec:threshold}), extensions of threshold signatures usually require more than secret sharing. For unweighted threshold signatures (in section \ref{sec:weighted}) and aggregated signatures (in subsection \ref{subsec:aggregate}), SNARK is a useful cryptographic primitive~\cite{DBLP:conf/ccs/DasCXNB023}. SNARK stands for \textit{Succinct Non-interactive ARgument of Knowledge}. This subsection firstly introduce SNARK briefly and then define it. 

SNARK is motivated in the verifiable computation setting. Suppose a person, Alice, wants to evaluate a function $f:\{0,1\}^{s}\to \{0,1\}^t$ on an input $x\in \{0,1\}^s$. However she does not want to do the computation herself. Therefore, Alice asks another person, Bob, to compute $f(x)$ for her. After a while, Bob tells Alice the result is $y$. Alice does not trust Bob and wants to verify if $y$ is really $f(x)$. If verificating $y=f(x)$ is easier than evaluating $f(x)$, Alice can reduce her own workload. Informally, SNARK allows Bob to succinctly argue that $y=f(x)$. 

\paragraph{Arithmetic circuit} For a field $\mathbb{F}$, an $\mathbb{F}$-\textit{arithmetic circuit} takes inputs that are elements in $\mathbb{F}$, and outputs elements in $\mathbb{F}$. We can natually use an arithmetic circuit to implement the function $f$ that Alice wants to evaluate. 

\paragraph{Circuit Satisfiability Problem} While it is feasible to design an argument system for the arithmetic circuit of $f$, the argument might not be succinct. Therefore, usually we first transform the problem of checking circuit evaluation $y=f(x)$ to the circuit satisfiability problem, such that $y=f(x)\iff \exists w, C(x, w)=y$. $w$ is called a \textit{witness} that the prover should find for a new verification circuit $C$. Particularly, we can also design a circuit $C'$ from $C$ such that $C(x,w)=y\iff C'((x,y), w)=0$, treating both $x$ and $y$ as input in $C'$.  Modern snarks are mostly designed for intermediate representations such as circuit satisfiability instances. 

\begin{definition}{(\textbf{Arithmetic Circuit Satisfiability Problem})}
    An \textit{arithmetic circuit satisfiability problem} of an $\mathbb{F}$-arithmetic circuit $C:\mathbb{F}^{k_x} \times \mathbb{F}^{k_w} \to \mathbb{F}^{k_y}$ is captured by the relation $\mathcal{R}_C = \{(x, w): C(x, w)=0^{k_y}\}$; its language is $\mathcal{L}_C=\{x\in \mathbb{F}^{k_x}: \exists w\in\mathbb{F}^{k_w}, (x, w)\in \mathcal{R}_C \}$.  
\end{definition}

A SNARK for circuit satisfiability problem in a field $\mathbb{F}$ is a triple of polynomial time algorithm $(\mathsf{KeyGen}, \mathsf{Prove}, \mathsf{Verify})$:
\begin{itemize}
    \item $\mathsf{KeyGen}(1^\lambda, C)\to (\mathsf{pk}, \mathsf{vk})$. On input a security parameter $\lambda$ and an $\mathbb{F}$-arithmetic circuit $C$, probabilistically generate a proving key $\mathsf{pk}$ and a verification key $\mathsf{vk}$. Both keys are published as public parameters, so that everyone can use them to prove or verify the membership of language $\mathcal{L}_C$. 
    \item $\mathsf{Prove}(\mathsf{pk}, x, w)\to \pi$. On input a proving key $\mathsf{pk}$ and a valid pair $(x, w)\in \mathcal{R}_C$, the prover outputs a non-interactive proof $\pi$ for the statement $x\in \mathcal{L}_C$. 
    \item $\mathsf{Verify}(\mathsf{vk}, x, \pi)\to b$. On input a verification key $\mathsf{vk}$, input $x$ and a proof $\pi$, the verifier outputs $b=1$ if he is convinced that $x\in \mathcal{L}_C$. 
\end{itemize}

A SNARK satisfies the following properties: 
\begin{itemize}
    \item \textbf{Completeness}: For every security parameter $\lambda$, any $\mathbb{F}$-arithmetic circuit $C$, and any $(x,w)\in\mathcal{R}_C$, the honest prover can convince the verifier. Namely, $b = 1$ with probability $1-negl(\lambda)$ in the following experiment: $(\mathsf{pk}, \mathsf{vk}) \to \mathsf{KeyGen}(1^{\lambda}, C); \pi \leftarrow \mathsf{Prove}(\mathsf{pk}, x, w); b \leftarrow \mathsf{Verify}(\mathsf{vk}, x, \pi)$.
    \item \textbf{Succinctness}. An honestly generated proof $\pi$ has $O_\lambda(1)$ bits and $\mathsf{Verify}($ $\mathsf{vk},x,\pi)$ runs in time $O_\lambda(|x|)$. (Here, $O_\lambda$ hides a fixed polynomial factor in $\lambda$.) Note that the evaluation time and the size of witness $w$ might be significantly larger than $|x|$. 
    \item \textbf{Proof of knowledge (and soundness)}. If the verifier accepts a proof output by a bounded prover, then the prover “knows” a witness for the given instance. (In particular, soundness holds against bounded provers.) Namely, for every $poly(\lambda)$-size adversary $\mathcal{A}$, there is a $poly(\lambda)$-size extractor $\mathcal{E}$ such that $\mathsf{Verify}(\mathsf{vk}, x, \pi) = 1$ and $(x, w) \in \mathcal{R}_C$ with probability $negl(\lambda)$ in the following experiment: $(\mathsf{pk}, \mathsf{vk}) \leftarrow \mathsf{KeyGen}(1^\lambda, C); (x, \pi) \leftarrow \mathcal{A}(\mathsf{pk}, \mathsf{vk}); a \leftarrow \mathcal{E}(\mathsf{pk}, \mathsf{vk})$.
\end{itemize}

\paragraph{Remark on zero-knowledge} Usually threshold signatures do not care about privacy of the subset of signers, therefore we do not use the zero-knowledge part, which (informally) requires that the proof $\pi$ does not leak any information about the witness $w$, of the well studied zk-SNARKs. 

A numerous number of SNARKs with acceptable concrete efficiency have been proposed in the literature, including Groth16~\cite{DBLP:conf/eurocrypt/Groth16}, BulletProofs~\cite{DBLP:conf/sp/BunzBBPWM18}, Plonk~\cite{DBLP:journals/iacr/GabizonWC19}, Aurora~\cite{DBLP:conf/eurocrypt/Ben-SassonCRSVW19}, etc. There are also many tools, libraries and compilers, such as gnark, that implement zk-SANRKs that allow programmers with little cryptography knowledge to develop zk-SNARKs applications for their own problems. 

Modern zk-SNARKs are mostly based on Interactive Oracle Proofs~\cite{DBLP:conf/tcc/Ben-SassonCS16} and implement the oracles with polynomial commitment schemes~\cite{DBLP:conf/asiacrypt/KateZG10}.  Constructions of SNARKs are much more complicated than digital signature schemes and secret sharing, so we do not present complete SNARK protocols in this survey. 


\newpage
This section formally defines threshold signatures and introduces a few classical constructions of threshold signatures. 
\subsection{Definition}
Threshold signature scheme involves $n$ signers. The typical setting is that any group of $t (1<t\le n)$ signers can collaborate to create a signature for a message $m$. However, any group of fewer than $t$ signers cannot create a signature for any messages. $t$ is the threshold. Each of the $n$ signers only possesses a share of the secret key, while one public key exists that can verify the validity of signatures. 

\paragraph{Trivial Construction} A trivial solution for threshold signature scheme is that each signer generates a secret/public key pair using a digital signature scheme and a concatenation of at least $t$ signatures is considered as a valid signature. The public key of this threshold signature scheme is the list of all $n$ public keys. The disadvantages of the na\"ive threshold signature scheme are (1) the signature size is $\Theta(t)$, (2) the size of public key is $\Theta(n)$ and (3) consequently the verification time is $\Theta(n)$. Note that usually we consider $t = \Theta(n)$. Ideally, we want threshold signature schemes with public keys/signatures of constant size and constant verification time. 

\begin{definition}
    A $(n, t)$-\textbf{threshold signature scheme}, $\mathsf{TS}$, consists of four algorithms $(\mathsf{Setup, Gen, Sig, Ver})$ defines as the following:
    \begin{itemize}
        \item $\mathsf{Setup}(1^\lambda)\to \mathsf{par}$ takes as input the security parameter $1^\lambda$ and outputs global public parameters $\mathsf{par}$, where $\mathsf{par}$ implicitly defines sets of public keys, secret keys, messages and signatures, and all algorithms related to $\mathsf{TS}$ implicitly take $\mathsf{par}$ as input. 
        \item $\mathsf{Gen(par)}\to (\mathsf{pk, sk_1,\dots, sk_n})$ takes as input public parameters $\mathsf{par}$, and outputs a public key $\mathsf{pk}$ and secret key shares $\mathsf{sk_1, \dots, sk_n}$. 
        \item $\mathsf{Sig}$ defines an interactive protocol among the a gourp of $t$ signers where signer $i$ owns the secret key share $\mathsf{sk_i}$. Given as input a message $\mathsf{m}$, these $t$ signers jointly output a signature $\sigma$. 
        \item $\mathsf{Ver}(\mathsf{pk}, \mathsf{m}, \sigma)\to b$ is deterministic, takes as input a public key $\mathsf{pk}$, a message $\mathsf{m}$, and a signature $\sigma$, and outputs a bit $b\in\{0,1\}$ indicating whether the signature is valid. 
    \end{itemize}
\end{definition}

\paragraph{Trusted or Distributed Key Generation} Usually the signers cannot choose their own secret key shares independently as if choosing a random secret key in single signer digital signature schemes. They have to collaborate to generate the secret key shares. To make sure that a secret key share of a signer is kept secret against other signers, we have to either assume a trusted third party acts as the dealer that distributes the shares to signers, or design a distributed protocol to securely generate the secret key shares. In this survey, we can assume that there are feasible solutions for distributed key generation.  

\paragraph{Interaction} In the $\mathsf{Sig}$ phase, $t$ signers might interact with each other. Some threshold signature schemes are interactive and others are non-interactive. Depending on the application scenarios, we might allow interaction among signers. In other cases, it is unrealistic for signers to interact with each other. In non-interactive threshold signature schemes, each signer $i$ generates a partial signature $\sigma_i$ for the message $\mathsf{m}$ using the secret key $\mathsf{sk_i}$. 

\subsection{Fault Tolerance} 
In the aforementioned setting of threshold signatures, we assume that the group of $t$ signers all honestly follow the protocol. In many real-world settings, a subset of signers might be malicious and deviate from the prescribed protocol in an attempt to prevent generating the desired signature correctly or generate signatures for other messages. We might tolerate up to $k$ maliciously corrupted signers such that any group of $\ge t'=t+k$ signers can guarantee the successful generation of valid signatures for any message, while a group of at most $t-1$ signers cannot create a valid signature for any message. 

\paragraph{Malicious Tolerance} {\color{red} TODO figure out the setting for malicious partial signatures}

\par In the next subsections, we present a few classical constructions of threshold signatures. Distributing the secret key shares among a group of participants that do not trust each other is a non-trivial task and rely on expensive Distributed Key Generation (DKG) protocols. DKG is not the focus of this survey, therefore our presentation simply assume that there is a trusted dealer that distributes all the secret key shares among the signers. In practice this trusted dealer is replaced with a suitable DKG protocol. 

\subsection{RSA Threshold Signatures}
The first construction of threshold signature scheme is based on RSA signatures according to~\cite{DBLP:conf/eurocrypt/Shoup00}. 

\paragraph{Setup} The dealer chooses at random two large primes of equal length $p$ and $q$, where $p=2p' + 1$, $q=2q'+1$, with $p'$, $q'$ themselves prime. Let $N=pq$ be the RSA modulus and make $N$ public. Let $M=p'q'$. Choose a cryptographic hash function $H$ that maps messages of arbitrary lengths to a number modulo $N$. 

\paragraph{Public Key} The dealer chooses a prime exponent $e$ such that $e>n$, and sets the public key $\mathsf{pk}$ as $(N, e)$. 

\paragraph{Secret Key Share} The dealer can compute $d=e^{-1}\pmod{M}$. Then the dealer chooses a random polynomial of degree at most $t'-1$ such that $a_0=d$ and all other coefficients $a_i (1\le i \le t'-1)$ are chosen independently at random. The polynomial is $f(X) = \sum_{i=0}^{t'-1}a_i X^i \in \mathbb{Z}[X]$. 

\par For $1\le i \le n$, the dealer computes the secret key share of signer $i$ as $\mathsf{sk_i} = f(i)\pmod M$. 

\paragraph{Verification Keys} Note that the secret and shares are integers in $\mathbb{Z}_M$, but $\mathbb{Z}_M$ is not a field because $M$ is composite number. Therefore, \cite{DBLP:conf/eurocrypt/Shoup00} designs additional verification keys in $Q_N$, a subgroup of squares in $Z_N^\times$. $Q_N$ is a cyclic group of order $m$. The dealer chooses a radnom $v\in Q_N$, and computes $v_i=v^{\mathsf{sk_i}}\in Q_N$ for all $i\in \{1, 2, \dots, n\}$. The verification key is $\mathsf{vk}=v$ and the verification key shares are $\mathsf{vk_i}=v_i$. 

\paragraph{Lagrange interpolation} Since $\mathbb{Z}_M$ is not a field, the reconstruction formula slightly deviates from standard Lagrange interpolation. Let $\Delta=n!$. For any subset of $S$ of $t'$ points in $\{0,1,\dots, n\}$, and for any $i\in\{0,1,\dots, n\}\setminus S$ and $j\in S$ , we can define: 
\begin{equation*}
    \lambda_{i,j}^S = \Delta \frac{\prod_{j'\in S\setminus\{j\}} (i-j')}{\prod_{j'\in S\setminus \{j\}} (j-j')}\in \mathbb{Z}
\end{equation*} 

The purpose of $\Delta$ is to make sure $\lambda_{i, j}^S$ are integers. Then, we have $\Delta\cdot f(i) = \sum_{j\in S} \lambda_{i,j}^S f(j)\pmod{M}$. 

\paragraph{Generating a signature share} For a message $M$, let $x=H(M)$. The signature share of player $i$ consists of $\sigma_i = x^{2 \Delta \mathsf{sk_i}}$. 

\paragraph{Combining Shares} Suppose we have valid shares from a set $S$ of players, where $S=\{i_1, \dots, i_k\}\subset \{1, 2, \dots, n\}$. To combine the shares, we compute $w=\prod_{j=1}^k \sigma_{i_j}^{2\lambda_{0, i_j}^S} = \prod_{j=1}^k x^{2\lambda_{0, i_j}^S \cdot 2\Delta \mathsf{sk_i}} =x^{4\Delta^2 d}\pmod {N}$. 

\paragraph{Verification of signature} To verify the signature $w$ using the public key $\mathsf{pk}=e$, it is suffice to check whether $w^e=x^{4\Delta^2}\pmod{N}$ holds. 


\subsection{Schnorr Threshold Signatures}

\subsection{BLS Threshold Signatures}
BLS threshold signatures are widely used in real-world applications. This subsection presents the BLS threshold signature initially proposed by Boldyreva \cite{DBLP:conf/pkc/Boldyreva03}, based on the BLS signature scheme. 

\subsection{Multi Signatures}

\subsection{Aggregators}

\subsection{Weighted Threshold Signatures}

\newpage
\section{Distributed Key Generation} \label{sec:dkg}

To deploy the threshold signature schemes in the distributed settings, such as in cryptocurrency applications, it is desirable to setup the public parameters and send the secret shares without trusting any third-party dealer. Instead, we prefer to share the secret using a distributed protocol, under a series of works on Distributed Key Generation (DKG). 

Shortly after threshold cryptosystems were proposed in the late 1980s~\cite{DBLP:conf/crypto/DesmedtF89}, early DKG protocols were developed~\cite{DBLP:conf/eurocrypt/Pedersen91a}. While threshold signature scheme itself is efficient where generating the threshold signature only requires linear $\Theta(n)$ total communication and the verification is even more efficient, the DKG protocol that sets up the keys for threshold signature can be quite expensive, requiring at least quadratic $\Omega(n^2)$ total communication. DKG indeed becomes the bottleneck for threshold signatures in volatile systems where the global signer set evolves rapidly, and hence has attracted many research efforts to improve the efficiency under various settings, for example~\cite{DBLP:conf/sp/DasYXMK022,DBLP:conf/uss/DasXK023}. 

Before introducing theses DKG protocols, we briefly sketch the ideas. With verifiable secret sharing, the participants can guarantee that the shares are correct and consistent, when someone shares a secret. However, they cannot simply ask one of the signers to take the role of the leader that chooses a secret by himself and share with others. The signers do not want anyone to know the secret $x$, and want to keep their own shares secret from all others. What if multiple signers jointly create this secret $x$? For example, signer $1$ and signer $2$ choose local secrets $a$ and $b$ respectively, then they share their secrets among all signers. Every signer $i$ receives $a_i$ and $b_i$. $a+b$ becomes a good secret such that every signer $i$ knows its share $a_i+b_i$, and this share of signer $i$ is unknown to others, as long as signer $1$ and $2$ do not collude with each other. 

\subsection{Pedersen's DKG for Threshold Cryptosystems}

\paragraph{Select keys} Let $\mathsf{commit}(x)$ denote a commitment to $x\in\{0,1\}^\ast$, with a slight abuse of notation that ignores the random string $r$. The keys are selected in the following rounds:

\begin{enumerate}
    \item Party $P_i$ chooses $s_i\in \mathbb{Z}_q$ at random and computes $h_i = g^{s_i}$. Then $P_i$ broadcasts $\mathsf{commit}(h_i)$ to all members. 
    \item When all $n$ members have broadcast a commitment, each $P_i$ opens his own commitment and reveals $h_i$. 
    \item The public key, $h$, is computed as $h = \prod_{i=1}^n h_i$. 
\end{enumerate}

Now all members know the public key $h$, but they do not know the secret key $s=\sum_{i=1}^n s_i$, unless they all work together. Therefore, $\{s_1, s_2,\dots, s_n\}$ should not be the secret shares for threshold signature schemes. Instead, each party $P_i$ will share his own secret $s_i$ among all members, using a verifiable secret sharing protocol in section \ref{subsubsec:vss}. 

\paragraph{Distribute keys} Each party $P_i$ chooses a polynomial $f_i$ of degree at most $t-1$ such that $f_i(0)= s_i$. $P_i$ sends $f_i(j)$ to party $P_j$ in a verifiable manner. After this phase, every party $P_j$ receives $(f_1(j), f_2(j), \dots, f_n(j))$. Note that the secret shares of $s=\sum_{i=1}^n s_i$ can be derived from the polynomial $f$ defined as $f(x)=f_1(x) + f_2(x) + \cdots + f_n(x)$. Every party $P_j$ can compute his share $f(j) = f_1(j)+f_2(j)+\cdots+f_n(j)$. Since $f$ has degree at most $t-1$, any $t$ parties can reconstruct the secret key $s=f(0)$ using Lagrange interpolation. 

With Pedersen's DKG, the secret key share of $P_i$ is defined as $\mathsf{sk}_i = f(i)=\sum_{j=1}^n f_j(i)$, for Schnorr and BLS threshold signatures. 

\subsection{Asynchronous DKG} 
Pederson's DKG is designed for synchronous networks, where messages are guaranteed to be delivered to the recipients within a known bounded delay. In real world scenarios, messages might be delayed arbitrarily long due to network issues or malicious nodes, which is modeled by an asynchronous network. Asynchronous networks, allow a bounded number $t_{A}$ of participants to be arbitrarily corrupted. Usually asynchronous networks tolerates $t_A = \Theta(n)$ faulty parties, and the asynchronous DKG protocols typically require at least $O(n^3)$ total communication, significantly more expensive than synchronous DKG protocols with $O(n^2)$ communication cost. 

\paragraph{Randomness Beacon, Threshold Cryptosystem and Asynchronous Consensus} The biggest challenge of DKG for threshold signatures is to reach consensus on which secret shares to aggregate for the final secret key $\mathsf{sk}$. Reaching agreement under asynchrony requires shared randomness. However, existing efficient mechanisms to generate shared randomness assume threshold secret-shared keys, hence creating a circularity. For our asynchronous DKG for threshold signatures, we should try to efficiently generate shared randomness internally instead of resorting to external random beacons. 

\paragraph{Technical overview of \cite{DBLP:conf/sp/DasYXMK022}} \cite{DBLP:conf/sp/DasYXMK022} presents an asynchronous DKG protocol to bootstrap threshold signature schemes. While Byzantine consensus protocols require shared randomness, Byzantine reliable broadcast protocols do not. Therefore, after the secret sharing step, \cite{DBLP:conf/sp/DasYXMK022} does not reach consensus on the secret shares immediately, but ask each party $P_i$ to broadcast a set $T_i$, called a \textit{intermediate key set}. The intermediate key set is later used in asynchronous Byzantine binary agreement and the secrets of the set act as the shared randomness when required by the consensus protocol. 

\begin{definition} (\textbf{Reliable Broadcast}). A protocol for a set of nodes $\{1,2,\dots, n\}$, where a distinguished node called the broadcaster holds an initial input $M$, is a reliable broadcast (RBC) protocol, if the following properties hold:
    \begin{itemize}
        \item \textit{Agreement}: If an honest node outputs a message $M'$ and another honest node outputs $M''$, then $M'=M''$. 
        \item \textit{Validity}: If the broadcaster is honest, all honest nodes eventually output the message $M$. 
        \item \textit{Totality}: If an honest node outputs a message, then every honest node eventually outputs a message.  
    \end{itemize}
\end{definition}

The DKG of \cite{DBLP:conf/sp/DasYXMK022} uses a \textit{validated} RBC protocol of \cite{DBLP:conf/ccs/DasX021}. To broadcast a message $M$, the communication cost is $O(n|M| + \kappa n^2)$, where $|M|$ is the size of $M$ and $\kappa$ is the output size of a collision-resistant hash function.  

To fit the asynchronous setting, \cite{DBLP:conf/sp/DasYXMK022} characterizes the requirement of secret sharing as \textit{asynchronous complete secret sharing (ACSS)} use the ACSS protocol of \cite{DBLP:conf/ccs/DasX021}. We omit the definition, requirement and constructions in this survey. 

\begin{definition} (\textbf{Asynchronous Byzantine Agreement (ABA)}). 
    A protocol for a set of nodes $\{1,\dots, n\}$, each node $P_i$ holding an initial binary input $b_i\in\{0,1\}$, is an ABA protocol, if the following properties hold under asynchrony:
    \begin{itemize}
        \item \textit{Agreement}: No two honest nodes output different values. 
        \item \textit{Validity}: If all honest nodes have the same input value, no honest node outputs a different value. 
        \item \textit{Termination}: Every honest node eventually outputs a value. 
    \end{itemize}
\end{definition}

\cite{DBLP:conf/sp/DasYXMK022} uses the ABA protocol of Crain~\cite{DBLP:journals/corr/abs-2002-08765} that provides an additional property called \textit{Good-Case-Coin-Free}. This property states that if all honest nodes input the same value to the ABA, then all honest nodes output without invoking the common coin (randomness).  The \textit{Good-Case-Coin-Free} property is required to ensure the liveness of the DKG protocol in case a malicious node does not propose any intermediate key set. With no intermediate key set, there is no shared randomness for the particular node. However, all honest nodes will have the same default input $0$, so that randomness is not required to achieve consensus. 


\paragraph{Performance} The expected total \textit{communication} cost of the ADKG protocol is $O(n^3)$. The expected \textit{computation} cost per node is $O(\kappa n^3)$, measured in number of elliptic curve exponentiations. The protocol terminates in $P(\log{n})$ \textit{rounds} in expectation. 

\newpage
\section{Succinct Weighted Threshold Signatures} \label{sec:weighted}

In the threshold signature schemes introduced in section \ref{sec:threshold}, every signer has the same unit weight. These schemes have constant signature size and verification key size, constant verification time. When we consider the application of voting in cryptocurrencies, different participants have different amounts of influence on the voting result. In Proof-of-Stake, the weight of a user is proportional to the amount of coins that he deposits as his stake. 

\paragraph{Simple Solution} A simple solution is to allow a user with a large weight to control multiple signers. In other words, it uses a \textit{virtualization} of threshold signature schemes. For example, in a system of 3 users, $U_1$ with a weight $1$, $U_2$ with a weight $2$ and $U_3$ with a weight $3$. The resulting threshold system consists of $6$ signers $\{P_1,P_2,\dots, P_6\}$. $U_1$ controls $S_1=\{P_1\}$, by owning the secret share of $P_1$. $U_2$ controls $S_2 = \{P_2,P_3\}$, while $U_3$ controls $S_3 = \{P_4, P_5, P_6\}$. Whenever a user wants to help create a signature, he asks all signers that he controls to participate the signing protocol. The simple solution is undesirable in PoS cryptocurrencies, because the stake (hence the weights) of accounts differ a lot. The ``richest'' account might possess more than $10^8$ times of another account. Then one rich account alone occupies at least $10^8$ secret key shares in the threshold signature scheme, which makes the virtualization solution practically infeasible. 


\paragraph{Sampling-based approach} To reduce the complexity of the aforementioned virtualization approach, Chaidos and Kiayias present a sampling based weighted threshold signature scheme~\cite{DBLP:journals/iacr/ChaidosK21}. The idea is to sample a subset of signers according to the weight distribution and let the sampled signers join an unweighted threshold signature scheme. However, sampling random subsets avoids the problem of designing efficient weighted threshold signature scheme rather then solves the problem. It introduces sampling bias, such that the actual threshold is not exact. \cite{DBLP:conf/ccs/DasCXNB023} further argues that the sampling-based approach requires a secure sample mechanism and is typically vulnerable to adaptive corruption. 

\paragraph{Weighted Secret Sharing} Another approach is to modify the secret sharing part in unweighted threshold signature schemes to accommodate the weighted setting. Generic weighted secret sharing (WSS) was first characterized by Beimel~\cite{DBLP:conf/tcc/BeimelTW05} and require that the share size of a signer is sublinear in his weight. Subsequent works develop more WSS schemes, such as \cite{DBLP:conf/crypto/GargJMSWZ23}. However, these schemes have undesirable concrete performance so far. 

\paragraph{SNARK} SNARKs allow efficiently verifying a general class of computation and are suitable solutions for weighted threshold signatures. Recall that SNARKs can generate a proof of $O_\lambda(1)$ size and $O_\lambda(|x|)$verification time for a circuit satisfiability problem $\exists w\in\{0,1\}^{k_w}, C(x, w)=0^{k_y}$. The most important goal is to allow anyone to efficiently verify that a threshold signature is valid or not. Therefore, we should reduce the size of $x$ as much as possible. If $|x| = O(1)$, the verification time is also constant. On the signer side, if we need $O(n)$ signers to generate a threshold signature, at least $O(n)$ bits of total communication are required. Ideally we want to control the overall communication/computation to generate a threshold signature $\sigma$ to be close to the lower bound $O(n)$. These goals can be achieved by SNARKs. 

\subsection{Weighted Threshold Signatures from Generic SNARKs}
\paragraph{Problem Setting} There are $n$ signers $\{P_1, P_2, \dots, P_n\}$, with weights $\{w_1,w_2,\dots, w_n\}\in \mathbb{Z}$ respectively, such that $\sum_{i=1}^n w_i = W$. Suppose we use BLS signature scheme, with a blinear pairing $e:\mathbb{G}_1\times \mathbb{G}_2\to \mathbb{G}_T$, a hash function $H:\{0,1\}^\ast\to \mathbb{G}_1$, a generator $g_2$ of prime order $p$ in group $\mathbb{G}_2$. Each signer $P_i$ has a secret key $\mathsf{sk}_i = x_i \in \mathbb{Z}_p$ and publishes a public key $\mathsf{pk}_i = g_2^{x_i}$. Let $\mathbf{pk}$ denote the vector $[\mathsf{pk}_1,\mathsf{pk}_2,\dots, \mathsf{pk}_n]$ and $\mathbf{w}$ denote the vector $[w_1,w_2,\dots, w_n]$. We consider a session of threshold signature for a message $m$, with a threshold requirement that can be arbitrary within $(0, W]$ and might vary in different sessions. A threshold signature $\tilde{\sigma}$ is valid is $\sum_{i\in S}w_i\ge t$, where $S$ is the signer set. SNARK-based threshold signature can allow the signature creator to arbitrarily specify the overall weights $t$ of the signer set $S$, different from secret sharing based approach where the threshold is fixed before aggregation. 

\paragraph{Remark about accessing the public keys} We aim to design a succinct weighted threshold signature scheme where the verification time is sublinear in $n$, the number of signers, or ideally does not grow with $n$. However, a verifier has to know all $n$ public keys. To circumvent this issue, we either (1) allow verifiers to learn these $n$ keys and possibly do preprocessing, and only care about the verification complexity for each single threshold signature after the preprocessing finishes, or (2) assume that the system is well integrated with SNARKs so that the verifier does not need to know all $n$ public keys, but has an oracle access to the public keys via a polynomial commitment scheme. From now on, we assume that the verifier knows a commitment $c_{\mathbf{pk}}$ of all public keys. Note that the commitment size $|c_{\mathbf{pk}}|$ is constant. Similarly the verifier also knows and accepts a commitment of the weightes: $c_{\mathbf{w}}$. 

\paragraph{Protocol} For a message $m$, each signer $P_i$ from a signer set $S$ creates a partial signature $\sigma_i = H(m)^{x_i}\in \mathbb{G}_1$. One combiner, who can be one of the signers in $S$ or any third party, receives all partial signatures $\sigma_S = \{\sigma_i\}_{i\in S}$, checks they are valid partial sigantures using their corresponding public keys and combines all the partial signatures by producing a succinct non-interactive proof that can be verified by everyone with access to the public information $(H(m), c_{\mathbf{pk}}, c_{\mathbf{w}})$. 

\paragraph{Circuit Satisfiability Problem for Weighted Threshold Signatures} Suppose the signer set is $S\subseteq \{1,2,\dots, n\}$. If  The relation that characterizes a valid signature is the following:

\begin{equation*}
    \mathcal{R}_{WTS} = 
\Bigg\{\space \begin{matrix}
         x = (m, t, c_{\mathbf{pk}, c_{\mathbf{w}}}, c_{S}, c_{\sigma_{S}})\\
         w = (\mathbf{pk}, \mathbf{w}, S, \sigma_S)\\
    \end{matrix}\quad \Bigg| \quad \begin{matrix}
        \mathbf{pk}\in \mathbb{G}_2^n; c_{\mathbf{pk}} = \mathsf{commit}(\mathbf{pk})\\
        \mathbf{w}\in \mathbb{F}^n, ||\mathbf{w}||_1<|\mathbb{F}|;  c_{\mathbf{w}} = \mathsf{commit}(\mathbf{w})\\
        S\subseteq \{1,2,\dots, n\}; c_{S} = \mathsf{commit}(S)\\
        \sigma_{S} \in \mathbb{G}_1^{|S|} ;c_{\sigma_{S}} = \mathsf{commit}(\sigma_S) \\
        \forall i\in S, e(H(m), \mathsf{pk}_i) = e(\sigma_i, g_2) \\
        t = \sum_{i\in S} w_i  \\
    \end{matrix}\space \Bigg\}
\end{equation*}

\paragraph{Remark on $||w||_1$} The maximum possible overall weight is $||w||_1 = \sum_{i=1}^n w_i$. SNARKs require that the number can be represented by a field element, so we require that $||w||_1<|\mathbb{F}|$. Therefore verifier complexity of SNARKs w.r.t. the $||w||_1$ can be $O(poly(\log(||w||)))$, which is significantly more efficient than $O(||w||_1)$, the complexity of the trivial virtualization approach. 

\paragraph{Signature size} The generated threshold signature is $x$ and a SNARK proof that $(x,w)\in \mathcal{R}_{WTS}$. Both parts consist of $O(1)$ bits. The verification time is also constant, when using SNARKs like Groth16 and Plonk. 

\subsection{Weighted Threshold Signatures from Special SNARKs}
While generic snarks based approaches create threshold signatures with very small signature size and verification time, \cite{DBLP:conf/ccs/DasCXNB023} argues that the aggregation time (prover complexity) is too high for practical applications. To optimize the aggregation time, \cite{DBLP:conf/ccs/DasCXNB023} proposed a weighted threshold signature scheme based on pairing based multisignature (introduced in section \ref{subsubsec:multisig}) and inner product arguments (IPA). We briefly introduce this scheme in the survey. 

\paragraph{Relation of IPA-based WTS} Let $S$ be the subset of signers who contribute partial signatures. Let $\mathbf{b}=[b_1,b_2,\dots, b_n]\in \{0,1\}^n$ be a bit vector where $b_i=1$ for each $i\in S$ and 0 otherwise. The subset multisignature on $m$ is $(\mathbf{b}, \sigma)$, where $\sigma = \prod_{i\in S}\sigma_i$. Let $\mathbf{\sigma}=[\sigma_1,\sigma_2,\dots, \sigma_n]$, where $\sigma_i$ can be an arbitrary element in $\mathbb{G_1}$ if no partial signature is received from signer $i$. Compared to the previous generic snark based approach, the aggregator $\mathcal{P}$ also computes the aggregated public key of the set $S$ as $\mathsf{pk}_S = \prod_{i\in S}\mathbf{pk}_i$ and the commitment $c_{\mathbf{b}}$ for the bit vector $\mathbf{b}$.  $\mathcal{P}$ sends the tuple $(m, c_{\mathbf{b}}, \mathsf{pk}_S, t, \sigma)$ to the verifier $\mathcal{V}$, along with a proof $\pi$ that these values are computed correctly, which is specified in the relation $\mathcal{R}_{IPA-WTS}$ below. After the SNARK check passes, $\mathcal{V}$ accepts the signature $\sigma$ if $e(H(m), \mathsf{pk}_S)  = e(\sigma, g_2)$. Assume that the two base groups for bilinear pairing are the same, i.e., $\mathbb{G}_1=\mathbb{G}_2 = \mathbb{G}$ and $g_1=g_2=g$. 

\begin{equation*}
    \mathcal{R}_{IPA-WTS} = 
\Bigg\{\space \begin{matrix}
         (c_{\mathbf{b}}, \mathsf{pk}_S, \sigma))\in \mathbb{G}^3\\
    \end{matrix}\quad \Bigg| \quad \begin{matrix}
        \mathbf{pk}\in \mathbb{G}^n; c_{\mathbf{pk}} = \mathsf{commit}(\mathbf{pk})\\
        \mathbf{w}\in \mathbb{F}^n, ||\mathbf{w}||_1<|\mathbb{F}|;  c_{\mathbf{w}} = \mathsf{commit}(\mathbf{w})\\
        \mathbf{b} \in \{0,1\}^n; c_{\mathbf{b}} = \mathsf{commit}(\mathbf{b}) \\
        \mathbf{\sigma}\in \mathbb{G}^n; <\mathbf{\sigma}, \mathbf{b}>=\sigma\\
        <\mathbf{w}, \mathbf{b}> = t; <\mathbf{pk}, \mathbf{b}>=\mathsf{pk}_{\mathbf{b}}\\
    \end{matrix}\space \Bigg\}
\end{equation*}

\paragraph{$\mathcal{R}_{IPA-WTS}$ as an IPA} The key idea of \cite{DBLP:conf/ccs/DasCXNB023} is to formulate $\mathcal{R}_{IPA-WTS}$ as an inner product argument (IPA) between $\mathcal{P}$ and $\mathcal{V}$. The constraints $t=<\mathsf{w},\mathsf{b}>$ and $\mathsf{pk}_{\mathbf{b}} = <\mathbf{pk}, \mathbf{b}>$ are naturally inner product constraints. Proving $\mathbb{b}$ is binary vector can also be solved by IPA~\cite{DBLP:conf/sp/BunzBBPWM18}. Among these inner product constraints, $<\mathbf{w}, \mathbf{b}>$ is simpler because both vectors consists of field elements and can be solved by existing IPA~\cite{DBLP:conf/asiacrypt/CampanelliNRZZ22}. $<\mathbf{pk}, \mathbf{b}>$ is the main challenge, because $\mathbf{pk}$ consists of group elements. The IPA scheme for group elements are the structured key generalized inner product argument (GIPA) proposed in \cite{DBLP:conf/asiacrypt/BunzMMTV21,DBLP:conf/tcc/Lee21}. \cite{DBLP:conf/ccs/DasCXNB023} argues that the GIPA approach has bad concrete efficiency, because they operate with the target group $\mathbb{G}_T$ directly. 

% \paragraph{IPA of \cite{DBLP:conf/asiacrypt/CampanelliNRZZ22} for $<\mathbf{x}, \mathbf{b}>$} The inner product $<\mathbf{pk}, \mathbf{b}> = g^{\sum_{i=1}^n b_i x_i}$. Let $\mathbf{x}=[x_1, x_2,\dots, x_n]$, then $<\mathbf{pk}, \mathbf{b}>=g_2^{<\mathbf{x}, \mathbf{b}>}$. The IPA protocol uses a powers-of-tau of degree $n$, i.e., $[g, g^\tau, g^{\tau^2},\dots, g^{\tau^n}]$, as the common reference string (CRS), where $\tau$ is not known to anyone. 

% {\color{red} TODO write more details, depending on availability}

The detailed construction is omitted in the survey. 

\newpage
For blockchain applications. Bitcoin wallet. 

Dynamic participation. 



\newpage
\section{Future Work: Incremental DKG} \label{sec:future-dkg}

Inspired by the application of blockchains, where the network participants are dynamic in the sense that (1) new participants can join and existing participants might not be active, (2) the weights of participants might evolve over time as their stake changes. In case we run key generation protocols very frequently because of the aforementioned dynamic updates, it is meaningful to design suitable DKG protocol that has good amortized complexity that outperforms the trivial approach to rerun the complete DKG protocol again. 

\paragraph{For SNARK-based threshold signature} Threshold signature schemes based on SNARKs do not maintain secret shares of keys, so that DKG is not required for setting up the keys. However, special SNARKs like \cite{DBLP:conf/ccs/DasCXNB023} require the aggregator to interact with signers and do precomputation to improve the efficiency per signature. It is interesting to reduce the amount of precomputation when the set of signers changes. 

\subsection{For secret-sharing based threshold signatures} 

\paragraph{Adding a member} There is a straightforward solution if we want to add a new member $P_{n+1}$ to the group, while keeping the threshold $t$ unchanged. This is because the secret key share of $P_{n+1}$ is nothing but $f(n+1)$, where $f$ is the polynomial such that $f(0)$ is the secret key and $f(i)$ is the secret key share of $P_i$ for $i\in\{1,\dots, n\}$. At least $t$ existing members are required to send to $P_{n+1}$ their reconstruction shares for $f(n+1)$. The secret shares of existing members remain the same. 

\paragraph{Removing a member} 
Removing an existing member is more complicated than adding a new member. If we want to keep the same secret key $f(0)=x$ and threshold $t$, the shares of remaining members must change. Suppose the member to remove is $P_n$, the remaining members should replace $f$ with another function $f'$ such that $f'(0)=f(0)=x$ but $f'(n)$ is indistinguishable from uniform randomness in the view of $P_n$. Since $f'$ and $f$ are different univariate polynomials of degree at most $t-1$, the valuation of $f'$ and $f$ coincide with at most $t-1$ locations, indicating that at least $(n-1)-(t-2) = n-t+1 = \Theta(n)$ remaining members must update their secret key shares. If there is no specific optimization, this is close to rerun the DKG protocol again, such as in \cite{DBLP:journals/npa/NoackS09}. 

Considering these challenges, it is interesting to design novel threshold signature schemes and DKG protocols that have desirable amortized complexity in a dynamic system where members might join or leave. 



\newpage
\section{Future Work: Alternative Solutions to Multi-Threshold Threshold Cryptosystem}

Traditional secret-sharing based threshold cryptosystem usually has a fixed threshold $t$. Researchers proposed to include dummy parties for flexible threshold. However, these schemes~\cite{DBLP:conf/crypto/DelerableeP08} require a threshold signature to specify the set of signers $S$ that generate the signature. In the verification step, the verifiers must know the set $S$, so that the verification time is $O(n)$. 

A direction that is worth exploring is to find a new light-weight cryptographic primitive for dynamic-threshold threshold signatures that allows (1) generating a signature of sublinear/constant size, (2) verifying any signature in sublinear/constant time using a verification key of sublinear/constant size. By light-weight, it cannot use complicated primitives like SNARKs. I conject that there is a solution. The solution should be highly related to secret sharing, but cannot use discrete log based cryptography.  

\newpage
\section{Conclusion}
This survey presents classical and SNARK-based threshold signature schemes. Threshold signatures are essential in distributed systems and cryptocurrencies. Although research on threshold signatures started from more than 30 years ago, there are challenges that are not addressed enough, some of which arising from the latest emgerging application scenarios. There are a large body of works on threshold signatures, so that I cannot cover every aspect of threshold signature research, nor cover the details of latest relevant protocols.  However, this survey covers the components and constructions of classical threshold signatures, mentions other related terms that might be useful for interested readers. As for future research, this survey suggests a few promising directions, including improving aggregate signatures using SNARKs, designing efficient incremental DKG and ambitiously proposing a new cryptographic paradigm for threshold signatures with an arbitrary threshold. 


% \begin{theorem}
% This is a sample theorem. The run-in heading is set in bold, while
% the following text appears in italics. Definitions, lemmas,
% propositions, and corollaries are styled the same way.
% \end{theorem}

\newpage

\bibliographystyle{splncs04}
\bibliography{reference}

\end{document}
