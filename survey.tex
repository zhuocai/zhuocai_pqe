\documentclass[runningheads]{llncs}
%
\usepackage[T1]{fontenc}
% T1 fonts will be used to generate the final print and online PDFs,
% so please use T1 fonts in your manuscript whenever possible.
% Other font encondings may result in incorrect characters.
%
\usepackage{graphicx}
\usepackage{amsmath}
\usepackage{amssymb}
\usepackage{xcolor}
% Used for displaying a sample figure. If possible, figure files should
% be included in EPS format.
%
% If you use the hyperref package, please uncomment the following two lines
% to display URLs in blue roman font according to Springer's eBook style:
%\usepackage{color}
%\renewcommand\UrlFont{\color{blue}\rmfamily}
%\urlstyle{rm}
%
\begin{document}
%
\title{Threshold Signatures and Applications in Blockchains}
%
%\titlerunning{Abbreviated paper title}
% If the paper title is too long for the running head, you can set
% an abbreviated paper title here
%
\author{Zhuo Cai}
%
% \authorrunning{F. Author et al.}
% First names are abbreviated in the running head.
% If there are more than two authors, 'et al.' is used.
%
\institute{Hong Kong University of Science and Technology}
%
\maketitle              % typeset the header of the contribution
%
\begin{abstract}
The abstract should briefly summarize the contents of the paper in
150--250 words.
\end{abstract}
%
%

\tableofcontents

\section{Introduction}

\paragraph{Motivation of classical threshold signatures} The early works of threshold signatures are tightly related to the general concept of threshold cryptosystems. In companies, or committees, a group of members might jointly represent the authority, instead of granting one person absolute power. Therefore, threshold signatures are introduced where all $n$ members jointly maintain the secret key and only a subset of at least $t$ can collaborate to generate a valid signature on behalf of the group. For example, in a committee of 3 members, any member cannot unilaterally make a decision, but any combination of 2 members are allowed to make decisions as long as they collaborate. Threshold cryptosystems are also important in fault-tolerant distributed systems where any node might crash at some time. 

\paragraph{Classical threshold signatures} The most popular threshold signatures are based on Schnorr or BLS digital signature schemes and verifiable secret sharing~\cite{DBLP:journals/joc/BonehLS04,DBLP:conf/eurocrypt/Pedersen91a,DBLP:conf/crypto/CritesKM23}. The goal of threshold signature scheme is to be efficient. It should be much more efficient than the na\"ive solution of concatenating ordinary signatures, which create signatures of $\Theta(n)$ size and require $\Theta(n)$ time for verifiers. Threshold Schnorr signatures require signers to interact with each other for at least two rounds to generate signatures. In contrast, BLS threshold signature is non-interactive but requires additional cryptography assumption, since the underlying BLS signature requires bilinear pairing. They both achieve constant signature size, verification key size and verification time. 

\paragraph{Limitations} In light of new application scenarios, such as cryptocurrencies, the classical threshold signatures have the following limitations: (1) every signer has the same unit weight, (2) the threshold is fixed, (3) the setup step, i.e. distributed key generation (DKG), requires at least $O(n^3)$ communication in asynchronous setting in existing protocols, although the signature is efficient afterwards. SNARK (\textit{Succinct Non-interactive ARguments of Knowledge}), a cryptography primitive that has been close to practical in recent years, allows an alternative paradigm of threshold signatures in weighted and multi-threshold setting. However, SNARK is complicated and usually incurs large concrete cost even though the asymptotic complexity is comparable with classical threshold signatures. 

This survey presents the basic components of threshold signatures in section \ref{sec:basic}, including digital signature schemes, verifiable secret sharing and SNARKs. Section \ref{sec:threshold} presents classical threshold signatures based on RSA, Schnorr and BLS signatures. It also discusses related exotic signature schemes including multi-signature and aggregate signatures. Section \ref{sec:dkg} discusses DKG protocols, which is not only the necessary setup step of threshold signature scheme but also becomes the efficiency bottleneck in dynamic systems. More specifically, we introduce the classical Pedersen's DKG and briefly discuss a recent work on DKG in the challenging asynchronous setting. Next in section \ref{sec:weighted} we present a recent work that designs succinct threshold signatures in weighted and multi-threshold setting, using a specially designed SNARK. In sections \ref{sec:future-aggregate}-\ref{sec:multi-threshold}, we introduce a few possible future directions. 




\subsection{Cryptographic Assumptions}

\subsection{Pairing}

\subsection{Digital Signatures}
\begin{definition}{(Digital Signature)} A digital signature scheme, $SGN(KeyGen,$ $ Sign, Verify)$, consists of four algorithms defined as follows: 
\end{definition}
\begin{itemize}
    \item $(pk, sk)\leftarrow SGN.Setup(\kappa, pp)$: Given the security parameter $\kappa$ and public parameters $pp$, it generate a pair of public/secret keys $(pk, sk)$. 
    \item 
\end{itemize}

\subsection{Threshold Signatures}

\subsection{Multi Signatures}

\subsection{Aggregators}

\subsection{Weighted Threshold Signatures}

\subsection{Adaptive Security}


\section{Classical Threshold Signatures} \label{sec:threshold}

This section formally defines threshold signatures and introduces a few classical constructions of threshold signatures. 
\subsection{Definition}
Threshold signature scheme involves $n$ signers. The typical setting is that any group of $t (1<t\le n)$ signers can collaborate to create a signature for a message $m$. However, any group of fewer than $t$ signers cannot create a signature for any messages. $t$ is the threshold. Each of the $n$ signers only possesses a share of the secret key, while one public key exists that can verify the validity of signatures. 

\paragraph{Trivial Construction} A trivial solution for threshold signature scheme is that each signer generates a secret/public key pair using a digital signature scheme and a concatenation of at least $t$ signatures is considered as a valid signature. The public key of this threshold signature scheme is the list of all $n$ public keys. The disadvantages of the na\"ive threshold signature scheme are (1) the signature size is $\Theta(t)$, (2) the size of public key is $\Theta(n)$ and (3) consequently the verification time is $\Theta(n)$. Note that usually we consider $t = \Theta(n)$. Ideally, we want threshold signature schemes with public keys/signatures of constant size and constant verification time. 

\begin{definition}
    A $(n, t)$-\textbf{threshold signature scheme}, $\mathsf{TS}$, consists of four algorithms $(\mathsf{Setup, Gen, Sig, Ver})$ defines as the following:
    \begin{itemize}
        \item $\mathsf{Setup}(1^\lambda)\to \mathsf{par}$ takes as input the security parameter $1^\lambda$ and outputs global public parameters $\mathsf{par}$, where $\mathsf{par}$ implicitly defines sets of public keys, secret keys, messages and signatures, and all algorithms related to $\mathsf{TS}$ implicitly take $\mathsf{par}$ as input. 
        \item $\mathsf{Gen(par)}\to (\mathsf{pk, sk_1,\dots, sk_n})$ takes as input public parameters $\mathsf{par}$, and outputs a public key $\mathsf{pk}$ and secret key shares $\mathsf{sk_1, \dots, sk_n}$. 
        \item $\mathsf{Sig}$ defines an interactive protocol among the a gourp of $t$ signers where signer $i$ owns the secret key share $\mathsf{sk_i}$. Given as input a message $\mathsf{m}$, these $t$ signers jointly output a signature $\sigma$. 
        \item $\mathsf{Ver}(\mathsf{pk}, \mathsf{m}, \sigma)\to b$ is deterministic, takes as input a public key $\mathsf{pk}$, a message $\mathsf{m}$, and a signature $\sigma$, and outputs a bit $b\in\{0,1\}$ indicating whether the signature is valid. 
    \end{itemize}
\end{definition}

\paragraph{Trusted or Distributed Key Generation} Usually the signers cannot choose their own secret key shares independently as if choosing a random secret key in single signer digital signature schemes. They have to collaborate to generate the secret key shares. To make sure that a secret key share of a signer is kept secret against other signers, we have to either assume a trusted third party acts as the dealer that distributes the shares to signers, or design a distributed protocol to securely generate the secret key shares. We delay the introduction of DKG protocols to section \ref{sec:dkg}. In this section, we can assume that there are feasible solutions for distributed key generation.  

\paragraph{Interaction} In the $\mathsf{Sig}$ phase, $t$ signers might interact with each other. Some threshold signature schemes are interactive and others are non-interactive. Depending on the application scenarios, we might allow interaction among signers. In other cases, it is unrealistic for signers to interact with each other. In non-interactive threshold signature schemes, each signer $i$ generates a partial signature $\sigma_i$ for the message $\mathsf{m}$ using the secret key $\mathsf{sk_i}$. 

\subsection{Fault Tolerance} 
In the aforementioned setting of threshold signatures, we assume that the group of $t$ signers all honestly follow the protocol. In many real-world settings, a subset of signers might be malicious and deviate from the prescribed protocol in an attempt to prevent generating the desired signature correctly or generate signatures for other messages. We might tolerate up to $k$ maliciously corrupted signers such that any group of $\ge t'=t+k$ signers can guarantee the successful generation of valid signatures for any message, while a group of at most $t-1$ signers cannot create a valid signature for any message. 

\paragraph{Malicious Tolerance} {\color{red} TODO figure out the setting for malicious partial signatures}

\par In the next subsections, we present a few classical constructions of threshold signatures. Distributing the secret key shares among a group of participants that do not trust each other is a non-trivial task and rely on expensive Distributed Key Generation (DKG) protocols. DKG is not the focus of this survey, therefore our presentation simply assume that there is a trusted dealer that distributes all the secret key shares among the signers. In practice this trusted dealer is replaced with a suitable DKG protocol. 

\subsection{RSA Threshold Signatures}
The first construction of threshold signature scheme is based on RSA signatures according to~\cite{DBLP:conf/eurocrypt/Shoup00}. 

\paragraph{Setup} The dealer chooses at random two large primes of equal length $p$ and $q$, where $p=2p' + 1$, $q=2q'+1$, with $p'$, $q'$ themselves prime. Let $N=pq$ be the RSA modulus and make $N$ public. Let $M=p'q'$. Choose a cryptographic hash function $H$ that maps messages of arbitrary lengths to a number modulo $N$. 

\paragraph{Public Key} The dealer chooses a prime exponent $e$ such that $e>n$, and sets the public key $\mathsf{pk}$ as $(N, e)$. 

\paragraph{Secret Key Share} The dealer can compute $d=e^{-1}\pmod{M}$. Then the dealer chooses a random polynomial of degree at most $t'-1$ such that $a_0=d$ and all other coefficients $a_i (1\le i \le t'-1)$ are chosen independently at random. The polynomial is $f(X) = \sum_{i=0}^{t'-1}a_i X^i \in \mathbb{Z}[X]$. 

\par For $1\le i \le n$, the dealer computes the secret key share of signer $i$ as $\mathsf{sk_i} = f(i)\pmod M$. 

\paragraph{Verification Keys} Note that the secret and shares are integers in $\mathbb{Z}_M$, but $\mathbb{Z}_M$ is not a field because $M$ is composite number. Therefore, \cite{DBLP:conf/eurocrypt/Shoup00} designs additional verification keys in $Q_N$, a subgroup of squares in $Z_N^\times$. $Q_N$ is a cyclic group of order $m$. The dealer chooses a radnom $v\in Q_N$, and computes $v_i=v^{\mathsf{sk_i}}\in Q_N$ for all $i\in \{1, 2, \dots, n\}$. The verification key is $\mathsf{vk}=v$ and the verification key shares are $\mathsf{vk_i}=v_i$. 

\paragraph{Lagrange interpolation} Since $\mathbb{Z}_M$ is not a field, the reconstruction formula slightly deviates from standard Lagrange interpolation. Let $\Delta=n!$. For any subset of $S$ of $t'$ points in $\{0,1,\dots, n\}$, and for any $i\in\{0,1,\dots, n\}\setminus S$ and $j\in S$ , we can define: 
\begin{equation*}
    \lambda_{i,j}^S = \Delta \frac{\prod_{j'\in S\setminus\{j\}} (i-j')}{\prod_{j'\in S\setminus \{j\}} (j-j')}\in \mathbb{Z}
\end{equation*} 

The purpose of $\Delta$ is to make sure $\lambda_{i, j}^S$ are integers. Then, we have $\Delta\cdot f(i) = \sum_{j\in S} \lambda_{i,j}^S f(j)\pmod{M}$. 

\paragraph{Generating a signature share} For a message $M$, let $x=H(M)$. The signature share of player $i$ consists of $\sigma_i = x^{2 \Delta \mathsf{sk_i}}$. 

\paragraph{Combining Shares} Suppose we have valid shares from a set $S$ of players, where $S=\{i_1, \dots, i_k\}\subset \{1, 2, \dots, n\}$. To combine the shares, we compute $w=\prod_{j=1}^k \sigma_{i_j}^{2\lambda_{0, i_j}^S} = \prod_{j=1}^k x^{2\lambda_{0, i_j}^S \cdot 2\Delta \mathsf{sk_i}} =x^{4\Delta^2 d}\pmod {N}$. 

\paragraph{Verification of signature} To verify the signature $w$ using the public key $\mathsf{pk}=e$, it is suffice to check whether $w^e=x^{4\Delta^2}\pmod{N}$ holds. 


\subsection{Schnorr Threshold Signatures}
Schnorr threshold signatures are based on single-party schnorr signature scheme. It is one of the popular practical threshold signature schemes, due to its simplicity. We firstly informally sketch the challenge of extending Schnorr signatures to threshold cryptosystem. Recall that Schnorr signature requires a signer to create a random number $k$ for one signature in the exponent. In the case where multiple signers create different random numbers, since the reconstruction phase of Shamir secret sharing requires addition operation on the secret shares, signers must achieve some consensus on the random numbers. Therefore, Schnorr threshold signature schemes usually require a few rounds of interaction among the signers to produce a valid signature. We introduce $\mathsf{Sparkle}$, a simple three-round threshold signature scheme presented in \cite{DBLP:conf/crypto/CritesKM23}.

\paragraph{Setup and Key Generation} The public parameters include a security parameter $1^\lambda$, a group $\mathbb{G}$ with a generator $g$ of prime order $p$, two hash function $\mathsf{H_{cm}}, \mathsf{H_{sig}}: \{0,1\}^\ast \to \mathbb{Z}_p$. The secret key $\mathsf{sk}=x$ is generated from $\mathbb{Z}_p$ and the corresponding public key is $\mathsf{pk}=y:=g^x$. The secret key shares are Shamir secret shares of $x$ among the $n$ signers with a threshold $t$. Signer $i$ receivers $\mathsf{sk_i}=x_i$. 

\paragraph{Signing Round 1} ($\mathsf{Sign}$) On input a message $m$ and a signing set $S$, each participant $i\in S$ samples a random number $r_i$ from $\mathbb{Z}_p$, computes $R_i=g^{r_i}$ and $\mathsf{cm}_i=\mathsf{H_{cm}}(m, S, R_i)$, and outputs their commitment $\mathsf{cm}_i$. 

\paragraph{Signing Round 2} ($\mathsf{Sign'}$) On input commitments $\{\mathsf{cm}_j\}_{j\in S}$, each participant $i\in S$ outputs their nonce $R_i$. 

\paragraph{Signing Round 3} ($\mathsf{Sign''}$) On input nonces $\{R_j\}_{j\in S}$, each participant $i\in S$ first checks that the commitments received in the first round are valid, i.e., $\mathsf{cm}_j=\mathsf{H_{cm}}(m, S, R_j)$ for all $j\ in S$. If not, return $\perp$. Else, each participant computes the aggregate nonce $\tilde{R} = \prod_{j\in S} R_j$, $c=\mathsf{H_{sig}}(y, m, \tilde{R})$, and partial signature $z_i = r_i + c\lambda_{S,i} x_i$, where $\lambda_{S,i}$ is the Lagrange coefficient for participant $i$ with respect to signing set $S$. Each participant outputs $z_i$. 

\paragraph{Combining Signatures} On input nonces $\{R_j\}_{j\in S}$ and partial signatures $\{z_j\}_{j\in S}$, the combiner computes $\tilde{R}=\prod_{j\in S} R_j$ and $z=\sum_{j\in S} z_j$, and outputs the signature $\sigma=(\tilde{R}, z)$. 

\paragraph{Verification} On input the joint public key $\mathsf{pk}=y$, a message $m$, and a purported signature $\sigma=(\tilde{R}, z)$, the verifier computes $c=\mathsf{H_{sig}}(y, m, \tilde{R})$, and accepts if $\tilde{R}y^c=g^z$. 

\paragraph{Security} \cite{DBLP:conf/crypto/CritesKM23} proves that $\mathsf{Sparkle}$ is secure under $t-1$ adaptive corruptions, under the one-more discrete log asssumption (AOMDL) in algebraic group models (AGM) and random oracle models (ROM).  

\subsection{BLS Threshold Signatures}
BLS threshold signatures are also widely used in real-world applications. In comparison to Schnorr threshold signatures, the signing algorithm of BLS signature scheme does not relies on local secret randomness, so the signers can delegate the task of combining partial signatures to any person and avoid the interaction among signers. This subsection presents the first BLS threshold signature initially proposed by Boldyreva \cite{DBLP:conf/pkc/Boldyreva03}, based on the BLS signature scheme. 

\paragraph{Setup and Key Generation} The setup and key generation steps are similar to Schnorr threshold signatures. However, BLS threshold signatures work under gap Diffie-Hellman assumption and require a bilinear pairing $e:\mathbb{G}_1\times \mathbb{G}_2\to \mathbb{G}_T$. 

\paragraph{Generating Partial Signatures} ($\mathsf{PartialSign}$) On input message $m$, a signer $i$ with secret key share $\mathsf{sk}_i = x_i$ can produce a partial signature as $\sigma_i = H(m)^{x_i}$. 

\paragraph{Combining Signatures} On input partial signatures $\{\sigma_i\}_{i\in S}$ such that $|S|=t$, the combiner computes $\sigma = \prod_{i\in S} \sigma_i^{\lambda_{S, i}}$, where $\lambda_{S,i}$ is the Lagrange coefficient that only depends on the signer set $S$ and every signer index $i$. 

\paragraph{Verification} On input a signature $\sigma$, a message $m$ and a public key $y=g^x$, the verifier accepts if $e(H(m), y) = e(\sigma, g)$. If the signature is valid, then $\sigma = H(m)^{\sum_{i\in S} \lambda_{S, i} \cdot x_i} = H(m)^x$. 


\subsection{Exotic Signatures}
Threshold signatures involve multiple signers. There are other variants of signature scheme that also involve multiple signers but have different functionalities. Exotic signatures refer to signatures with properties beyond the norm and functionalities of digital signature schemes. They are useful in different application scenarios. 

\subsubsection{Multi Signatures}
Multi signatures are threshold signatures with the restriction that the threshold $t$ is exactly $n$. A multi signature $\sigma$ on a message $m$ is valid only if all $n$ signers contribute to $\sigma$. 

Using BLS signature, suppose each signer $i$ has secret key $\mathsf{sk}_i = x_i\in \mathbb{Z}_p$ and publishes the public key $\mathsf{pk}_i = g_2^{x_i}\in \mathbb{G}_2$. On input a message $m$, the partial signature of signer $i$ is $H(m)^{x_i} \in \mathbb{G}_1$. From these partial signatures, the aggregated multisignature is defined as $\sigma = \prod_{i=1}^n \sigma_i$. The verification key for the multisignature is $\mathsf{pk} = \prod_{i=1}^n \mathsf{pk}_i$. To verify a multisignature $m$ using the verification key $\mathsf{pk}$, it suffices to check if $e(H(m), \mathsf{pk}) = e(\sigma, g_2)$. 

\paragraph{Comparison with BLS threshold signature} The verifier in the multisignature scheme must knows $\mathsf{pk}$ to efficiently verify the signature. $\mathsf{pk}$ depends on the signer set. In multisignature, the signer set is fixed, so everyone can precompute $\mathsf{pk}$ only once and the amortized cost is negligible. In threshold signatures, the signer set might be any subset $S$ consisting of $t$ signers of all $n$ signers. It is imfeasible to precompute $\mathsf{pk}_S$ for any possible $S$. 

\subsubsection{Group Signatures} 
A group signature scheme allows a member of a group to anonymously sign a message on behalf of the group. For example, it can be used by an employee of a large company where a verifier only cares about whether a message was signed by an employee, but does not care about which particular employee signed it. A group signature scheme has a group manager, who is in charge of adding group members and has the ability to reveal the original signer in the event of disputes. 

\subsubsection{Ring Signatures} 
Similar to group signatures, in a ring signature scheme, any signer $i$ can generate a valid signature on behalf of the whole signer set. There is no group manager and it is computationally difficult to determine the signer even for insiders. Ring signature was proposed by Rivest, Shamir and Tauman in the name of "how to leak a secret"~\cite{DBLP:conf/asiacrypt/RivestST01}, for the application that one member wants to leak a secret of a group but keep his identity secret completely. 







\subsection{Aggregate Signatures} \label{subsec:aggregate}
In threshold signatures, a group of $n$ signers try to create signatures for one common message $m$. Aggregate signatures consider the problem of aggregating $n$ signatures of $n$ different messages into a short signature. 

\subsection{BGLS Aggregate Signatures} \cite{DBLP:conf/eurocrypt/BonehGLS03} presents an aggregate signature scheme based on bilinear pairing. Their construction yields a constant size aggregated signature. 

\paragraph{Setup} The setup is similar to BLS signatures for $n$ disjoint signers. Choose a bilinear pairing $e:\mathbb{G}_1\times \mathbb{G}_2\to \mathbb{G}_T$, where the base groups $\mathbb{G}_1$ and $\mathbb{G}_2$ have prime order $p$ with generators $g_1$ and $g_2$ respectively. Every signer $i$ has a secret key $x_i\in \mathbb{Z}_p$ and the corresponding public key is $y_i = g_2^{x_i}\in \mathbb{G}_2$. All the public keys $y_i$ are publicly known. 

\paragraph{Signing} Each signer $i$ uses his secret key $x_i$ to sign a message $m_i$. The output is a BLG signature $\sigma_i = H(m)^{x_i}\in \mathbb{G}_1$.  

\paragraph{Aggregation} Upon receiving messages $\{m_i\}_{i\in S}$ and signatures $\{\sigma_i\}_{i\in S}$ from the aggregation subset of signers $S={s_1, s_2,\dots, s_k}$, the aggregator makes sure that the $k$ messages are distinct and computes $\sigma=\prod_{i=S} \sigma_i \in \mathbb{G}_1$. 

\paragraph{Aggregate Verification} Given a signer set $S$, the set of messages $\{m_i\}_{i\in S}$, the set of signatures $\{\sigma_i\}_{i\in S}$ and the public keys $\{y_i\}_{i\in S}$, the verifier accepts if 1) the messages are distinct and 2) $e(\sigma, g_2)=\prod_{i\in S}e(H(m_i), y_i)$. If the signature is valid, the second condition should satisfy because 
\begin{equation*}
    e(\sigma, g_2) = e(\prod_{i\in S}H(m_i)^{x_i}, g_2) = \prod_{i\in S} e(H(m_i), g_2)^{x_i} = \prod_{i\in S}e(H(m_i), g_2^{x_i}) %= \prod_{i\in S} e(H(m_i), y_i) 
\end{equation*}
{\color{red} TODO add some proof}

\paragraph{Efficiency} Although the size of aggregate signature is only constant, the verifier still has to spend $\Theta(n)$ time. {\color{red} TODO mention the application of SNARKs later}




\section{Distributed Key Generation}
%\frame{\tableofcontents[currentsection]}


\begin{frame}{DKG to setup threshold signatures}
    In BLS threshold signature, we assume that a trusted dealer choooses a random $x$ and distributes the shares among $\{P_1,\dots, P_n\}$. \\
    \begin{itemize}
        \item the dealer might distribute wrong shares! \\ \pause 
              - \textbf{verifiable} secret sharing: prove that shares/reconstruction are correct.   \pause 
        \item the dealer knows $x$. With $x$, he can create a signature by himself. \\ \pause
            - Remove the dealer, use a distributed key generation (DKG) protocol. 
    \end{itemize}
    \pause 
    \vspace{0.5em}
    Intuition of DKG: nobody should know $x$, then let $P_1$ select $s_1$ and $P_2$ select $s_2$, define $x=s_1+s_2$. \\ \pause 
    \vspace{0.5em}
    Issue: what if $P_1$ and $P_2$ collude with each other? \\ \pause 
    Solution: let every $P_i$ choose $s_i$, so that $x=s_1+s_2+\cdots s_n$. Every $P_i$ shares $s_i$ with others using VSS.  [Pedersen'91]
\end{frame}



\begin{frame}{Asynchronous DKG}
    Fault tolerant distributed protocols are expensive! Consider the cost of one node broadcasting a message to all nodes ($n$ is the number of all nodes and $t$ is the number of corrupted nodes): 
    \begin{itemize}
        \item In synchronous networks (messages are delivered with known bounded delays) ($n=2t+1$): $O(n^2)$ communication. 
        \item In asynchronous networks (messages are might be delayed arbitrarily long) ($n=3t+1$): $O(n^2)$ communication. 
    \end{itemize}
    ADKG complexity of ~\cite{DBLP:conf/sp/DasYXMK022}\footfullcite{DBLP:conf/sp/DasYXMK022}: $O(\kappa n^3)$ total communication, where $\kappa$ is the output size of a cryptographic hash function. \\
    ADKG might be the performance bottleneck of threshold signatures. \\
    \vspace{0.5em}
    Recent research tries to improve ADKG, that ideally requires $O(n^2)$ communication. 
    \end{frame}

\section{Succinct Weighted Threshold Signatures} \label{sec:weighted}

In the threshold signature schemes introduced in section \ref{sec:threshold}, every signer has the same unit weight. These schemes have constant signature size and verification key size, constant verification time. When we consider the application of voting in cryptocurrencies, different participants have different amounts of influence on the voting result. In Proof-of-Stake, the weight of a user is proportional to the amount of coins that he deposits as his stake. 

\paragraph{Simple Solution} A simple solution is to allow a user with a large weight to control multiple signers. In other words, it uses a \textit{virtualization} of threshold signature schemes. For example, in a system of 3 users, $U_1$ with a weight $1$, $U_2$ with a weight $2$ and $U_3$ with a weight $3$. The resulting threshold system consists of $6$ signers $\{P_1,P_2,\dots, P_6\}$. $U_1$ controls $S_1=\{P_1\}$, by owning the secret share of $P_1$. $U_2$ controls $S_2 = \{P_2,P_3\}$, while $U_3$ controls $S_3 = \{P_4, P_5, P_6\}$. Whenever a user wants to help create a signature, he asks all signers that he controls to participate the signing protocol. The simple solution is undesirable in PoS cryptocurrencies, because the stake (hence the weights) of accounts differ a lot. The ``richest'' account might possess more than $10^8$ times of another account. Then one rich account alone occupies at least $10^8$ secret key shares in the threshold signature scheme, which makes the virtualization solution practically infeasible. 


\paragraph{Sampling-based approach} To reduce the complexity of the aforementioned virtualization approach, Chaidos and Kiayias present a sampling based weighted threshold signature scheme~\cite{DBLP:journals/iacr/ChaidosK21}. The idea is to sample a subset of signers according to the weight distribution and let the sampled signers join an unweighted threshold signature scheme. However, sampling random subsets avoids the problem of designing efficient weighted threshold signature scheme rather then solves the problem. It introduces sampling bias, such that the actual threshold is not exact. \cite{DBLP:conf/ccs/DasCXNB023} further argues that the sampling-based approach requires a secure sample mechanism and is typically vulnerable to adaptive corruption. 

\paragraph{Weighted Secret Sharing} Another approach is to modify the secret sharing part in unweighted threshold signature schemes to accommodate the weighted setting. Generic weighted secret sharing (WSS) was first characterized by Beimel~\cite{DBLP:conf/tcc/BeimelTW05} and require that the share size of a signer is sublinear in his weight. Subsequent works develop more WSS schemes, such as \cite{DBLP:conf/crypto/GargJMSWZ23}. However, these schemes have undesirable concrete performance so far. 

\paragraph{SNARK} SNARKs allow efficiently verifying a general class of computation and are suitable solutions for weighted threshold signatures. Recall that SNARKs can generate a proof of $O_\lambda(1)$ size and $O_\lambda(|x|)$verification time for a circuit satisfiability problem $\exists w\in\{0,1\}^{k_w}, C(x, w)=0^{k_y}$. The most important goal is to allow anyone to efficiently verify that a threshold signature is valid or not. Therefore, we should reduce the size of $x$ as much as possible. If $|x| = O(1)$, the verification time is also constant. On the signer side, if we need $O(n)$ signers to generate a threshold signature, at least $O(n)$ bits of total communication are required. Ideally we want to control the overall communication/computation to generate a threshold signature $\sigma$ to be close to the lower bound $O(n)$. These goals can be achieved by SNARKs. 

\subsection{Weighted Threshold Signatures from Generic SNARKs}
\paragraph{Problem Setting} There are $n$ signers $\{P_1, P_2, \dots, P_n\}$, with weights $\{w_1,w_2,\dots, w_n\}\in \mathbb{Z}$ respectively, such that $\sum_{i=1}^n w_i = W$. Suppose we use BLS signature scheme, with a blinear pairing $e:\mathbb{G}_1\times \mathbb{G}_2\to \mathbb{G}_T$, a hash function $H:\{0,1\}^\ast\to \mathbb{G}_1$, a generator $g_2$ of prime order $p$ in group $\mathbb{G}_2$. Each signer $P_i$ has a secret key $\mathsf{sk}_i = x_i \in \mathbb{Z}_p$ and publishes a public key $\mathsf{pk}_i = g_2^{x_i}$. Let $\mathbf{pk}$ denote the vector $[\mathsf{pk}_1,\mathsf{pk}_2,\dots, \mathsf{pk}_n]$ and $\mathbf{w}$ denote the vector $[w_1,w_2,\dots, w_n]$. We consider a session of threshold signature for a message $m$, with a threshold requirement that can be arbitrary within $(0, W]$ and might vary in different sessions. A threshold signature $\tilde{\sigma}$ is valid is $\sum_{i\in S}w_i\ge t$, where $S$ is the signer set. SNARK-based threshold signature can allow the signature creator to arbitrarily specify the overall weights $t$ of the signer set $S$, different from secret sharing based approach where the threshold is fixed before aggregation. 

\paragraph{Remark about accessing the public keys} We aim to design a succinct weighted threshold signature scheme where the verification time is sublinear in $n$, the number of signers, or ideally does not grow with $n$. However, a verifier has to know all $n$ public keys. To circumvent this issue, we either (1) allow verifiers to learn these $n$ keys and possibly do preprocessing, and only care about the verification complexity for each single threshold signature after the preprocessing finishes, or (2) assume that the system is well integrated with SNARKs so that the verifier does not need to know all $n$ public keys, but has an oracle access to the public keys via a polynomial commitment scheme. From now on, we assume that the verifier knows a commitment $c_{\mathbf{pk}}$ of all public keys. Note that the commitment size $|c_{\mathbf{pk}}|$ is constant. Similarly the verifier also knows and accepts a commitment of the weightes: $c_{\mathbf{w}}$. 

\paragraph{Protocol} For a message $m$, each signer $P_i$ from a signer set $S$ creates a partial signature $\sigma_i = H(m)^{x_i}\in \mathbb{G}_1$. One combiner, who can be one of the signers in $S$ or any third party, receives all partial signatures $\sigma_S = \{\sigma_i\}_{i\in S}$, checks they are valid partial sigantures using their corresponding public keys and combines all the partial signatures by producing a succinct non-interactive proof that can be verified by everyone with access to the public information $(H(m), c_{\mathbf{pk}}, c_{\mathbf{w}})$. 

\paragraph{Circuit Satisfiability Problem for Weighted Threshold Signatures} Suppose the signer set is $S\subseteq \{1,2,\dots, n\}$. If  The relation that characterizes a valid signature is the following:

\begin{equation*}
    \mathcal{R}_{WTS} = 
\Bigg\{\space \begin{matrix}
         x = (m, t, c_{\mathbf{pk}, c_{\mathbf{w}}}, c_{S}, c_{\sigma_{S}})\\
         w = (\mathbf{pk}, \mathbf{w}, S, \sigma_S)\\
    \end{matrix}\quad \Bigg| \quad \begin{matrix}
        \mathbf{pk}\in \mathbb{G}_2^n; c_{\mathbf{pk}} = \mathsf{commit}(\mathbf{pk})\\
        \mathbf{w}\in \mathbb{F}^n, ||\mathbf{w}||_1<|\mathbb{F}|;  c_{\mathbf{w}} = \mathsf{commit}(\mathbf{w})\\
        S\subseteq \{1,2,\dots, n\}; c_{S} = \mathsf{commit}(S)\\
        \sigma_{S} \in \mathbb{G}_1^{|S|} ;c_{\sigma_{S}} = \mathsf{commit}(\sigma_S) \\
        \forall i\in S, e(H(m), \mathsf{pk}_i) = e(\sigma_i, g_2) \\
        t = \sum_{i\in S} w_i  \\
    \end{matrix}\space \Bigg\}
\end{equation*}

\paragraph{Remark on $||w||_1$} The maximum possible overall weight is $||w||_1 = \sum_{i=1}^n w_i$. SNARKs require that the number can be represented by a field element, so we require that $||w||_1<|\mathbb{F}|$. Therefore verifier complexity of SNARKs w.r.t. the $||w||_1$ can be $O(poly(\log(||w||)))$, which is significantly more efficient than $O(||w||_1)$, the complexity of the trivial virtualization approach. 

\paragraph{Signature size} The generated threshold signature is $x$ and a SNARK proof that $(x,w)\in \mathcal{R}_{WTS}$. Both parts consist of $O(1)$ bits. The verification time is also constant, when using SNARKs like Groth16 and Plonk. 

\subsection{Weighted Threshold Signatures from Special SNARKs}
While generic snarks based approaches create threshold signatures with very small signature size and verification time, \cite{DBLP:conf/ccs/DasCXNB023} argues that the aggregation time (prover complexity) is too high for practical applications. To optimize the aggregation time, \cite{DBLP:conf/ccs/DasCXNB023} proposed a weighted threshold signature scheme based on pairing based multisignature (introduced in section \ref{subsubsec:multisig}) and inner product arguments (IPA). We briefly introduce this scheme in the survey. 

\paragraph{Relation of IPA-based WTS} Let $S$ be the subset of signers who contribute partial signatures. Let $\mathbf{b}=[b_1,b_2,\dots, b_n]\in \{0,1\}^n$ be a bit vector where $b_i=1$ for each $i\in S$ and 0 otherwise. The subset multisignature on $m$ is $(\mathbf{b}, \sigma)$, where $\sigma = \prod_{i\in S}\sigma_i$. Let $\mathbf{\sigma}=[\sigma_1,\sigma_2,\dots, \sigma_n]$, where $\sigma_i$ can be an arbitrary element in $\mathbb{G_1}$ if no partial signature is received from signer $i$. Compared to the previous generic snark based approach, the aggregator $\mathcal{P}$ also computes the aggregated public key of the set $S$ as $\mathsf{pk}_S = \prod_{i\in S}\mathbf{pk}_i$ and the commitment $c_{\mathbf{b}}$ for the bit vector $\mathbf{b}$.  $\mathcal{P}$ sends the tuple $(m, c_{\mathbf{b}}, \mathsf{pk}_S, t, \sigma)$ to the verifier $\mathcal{V}$, along with a proof $\pi$ that these values are computed correctly, which is specified in the relation $\mathcal{R}_{IPA-WTS}$ below. After the SNARK check passes, $\mathcal{V}$ accepts the signature $\sigma$ if $e(H(m), \mathsf{pk}_S)  = e(\sigma, g_2)$. Assume that the two base groups for bilinear pairing are the same, i.e., $\mathbb{G}_1=\mathbb{G}_2 = \mathbb{G}$ and $g_1=g_2=g$. 

\begin{equation*}
    \mathcal{R}_{IPA-WTS} = 
\Bigg\{\space \begin{matrix}
         (c_{\mathbf{b}}, \mathsf{pk}_S, \sigma))\in \mathbb{G}^3\\
    \end{matrix}\quad \Bigg| \quad \begin{matrix}
        \mathbf{pk}\in \mathbb{G}^n; c_{\mathbf{pk}} = \mathsf{commit}(\mathbf{pk})\\
        \mathbf{w}\in \mathbb{F}^n, ||\mathbf{w}||_1<|\mathbb{F}|;  c_{\mathbf{w}} = \mathsf{commit}(\mathbf{w})\\
        \mathbf{b} \in \{0,1\}^n; c_{\mathbf{b}} = \mathsf{commit}(\mathbf{b}) \\
        \mathbf{\sigma}\in \mathbb{G}^n; <\mathbf{\sigma}, \mathbf{b}>=\sigma\\
        <\mathbf{w}, \mathbf{b}> = t; <\mathbf{pk}, \mathbf{b}>=\mathsf{pk}_{\mathbf{b}}\\
    \end{matrix}\space \Bigg\}
\end{equation*}

\paragraph{$\mathcal{R}_{IPA-WTS}$ as an IPA} The key idea of \cite{DBLP:conf/ccs/DasCXNB023} is to formulate $\mathcal{R}_{IPA-WTS}$ as an inner product argument (IPA) between $\mathcal{P}$ and $\mathcal{V}$. The constraints $t=<\mathsf{w},\mathsf{b}>$ and $\mathsf{pk}_{\mathbf{b}} = <\mathbf{pk}, \mathbf{b}>$ are naturally inner product constraints. Proving $\mathbb{b}$ is binary vector can also be solved by IPA~\cite{DBLP:conf/sp/BunzBBPWM18}. Among these inner product constraints, $<\mathbf{w}, \mathbf{b}>$ is simpler because both vectors consists of field elements and can be solved by existing IPA~\cite{DBLP:conf/asiacrypt/CampanelliNRZZ22}. $<\mathbf{pk}, \mathbf{b}>$ is the main challenge, because $\mathbf{pk}$ consists of group elements. The IPA scheme for group elements are the structured key generalized inner product argument (GIPA) proposed in \cite{DBLP:conf/asiacrypt/BunzMMTV21,DBLP:conf/tcc/Lee21}. \cite{DBLP:conf/ccs/DasCXNB023} argues that the GIPA approach has bad concrete efficiency, because they operate with the target group $\mathbb{G}_T$ directly. 

\paragraph{IPA of \cite{DBLP:conf/asiacrypt/CampanelliNRZZ22} for $<\mathbf{x}, \mathbf{b}>$} The inner product $<\mathbf{pk}, \mathbf{b}> = g^{\sum_{i=1}^n b_i x_i}$. Let $\mathbf{x}=[x_1, x_2,\dots, x_n]$, then $<\mathbf{pk}, \mathbf{b}>=g_2^{<\mathbf{x}, \mathbf{b}>}$. The IPA protocol uses a powers-of-tau of degree $n$, i.e., $[g, g^\tau, g^{\tau^2},\dots, g^{\tau^n}]$, as the common reference string (CRS), where $\tau$ is not known to anyone. 

{\color{red} TODO write more details, depending on availability}



\section{Topic 1}
\section{Future Work: Succinct Aggregate Signatures}

In section \ref{subsec:aggregate} we discussed an exotic signature scheme for aggregating signatures on different messages. However, in the BGLS aggregate signature scheme~\cite{DBLP:conf/eurocrypt/BonehGLS03}, the verifier should access all $n$ public keys for verification that takes $O(n)$ time. Using SNARKs, it is possible to aggregate multiple signatures on different messages into a succinct signature and allow efficient verification of the succinct signature using a short verification key. 

\subsection{Motivation from the Application of Democratic Voting} 

\paragraph{Traditional BFT protocols} In traditional distributed systems that consider fault tolerance, for example in a blockchain where less than half of participants can be malicious, it is not easy to achieve both consensus and liveness. Byzantine fault tolerant broadcast protocols have been proposed, in both synchronous setting and asynchronous setting. These protocols typically allow only one person to publish a message and achieve the goal that every honest participant receives the same message in the end, even at the presence of a certain ratio of malicious participants. 

\paragraph{Democratic Voting} We consider democratic voting in a distributed system. Creating blocks in blockchains is an important problem in the real world that can benefit from a democratic voting scheme,  where every cryptocurrency participant can publish a block proposal for any round. Democratic voting is different from the broadcast problem, because every participant, instead of only one participant, each time is allowed to propose his own message. The message with the most votes should be agreed by all honest participants in the end. Propagating the candidate messages across the network is relatively easy. The challenge is to propagate the votes and reach consensus on the voting result. Suppose every participant can vote for at most one candidate message. While it is theoretically possible for every participant $i$ to broadcast his vote to all others so that all honest participants reach consensus on the voting result, it is infeasible in large-scale applications, since an asynchronous byzantine broadcast session already requires $\Theta(n^2)$ total communication. 

\paragraph{Weighted threshold signature is not enough} While the weighted threshold signature schemes introduced in section \ref{sec:weighted} allow aggregating the votes for the same candidate, it is worth mentioning that the winning candidate $m_w$ might receive fewer than $n/2$ votes. Therefore, one threshold signature of the winning candidate with threshold $t<n/2$ alone cannot convince all honest participants that $m_w$ wins.  If a verifier in the future does not see the votes for other candidates, it is possible that there is another message candidate with more votes than $m_w$ in the view of the future verifier. Suppose $m_w$ receives $n/3$ votes and two other candidates $m_1$ and $m_2$ receive $n/4$ votes each, then the aggregate signature for $m_w$ is not sufficient to convince the future verifier but the aggregate signature for distinct candidates $(m_w, m_1, m_2)$ can convince verifiers that no message other than $m_w$ can receive more than $n/3$ votes, assuming that no participant casts $2$ or more votes. Note that there is alternative solution that does not require aggregate signature scheme for distinct messages, which is to ask all participants to vote again only for $m_w$ so that $m_w$ will by attested by a majority of votes. However, this solution might involve expensive communication to achieve consensus on $m_w$ among the currently active participants. 

\paragraph{Circuit Satisfiability Problem of Aggregate Signatures} Assume we want to aggregate the signatures for message $\{m_w, m_1, m_2,\dots, m_k\}$. Let $S_w$ denote the set of participants who vote for $m_w$ by broadcasting their partial signatures for $m_v$. Similarly, let $S_i$ denote the set of participants who vote for the message $m_i$. Let $\mathbf{S}$ denote $[S_1,\dots, S_k]$. Aggregating the signature for $m_w$ is exactly solved by a weighted threshold signature. Therefore, we focus on aggregating the signatures for all other messages and aim for succinct signatures that can be verified in time sublinear in $k$, preferably independent of $k$. Comparing to the WTS, the verifier also receives a succinct commitment of all messages. Let $\mathbf{m}$ denote $[m_1, m_2,\dots, m_k]$ and $c_{\mathbf{m}}$ denote its commitment. In our application of democratic voting, the verifier only cares about there exist messages that receive $t$ votes, but not about what these messages are exactly.  Note that the aggregate signature in section \ref{subsec:aggregate} requires that all $n$ messages are distinct, which is not assumed in this section. The relation of aggregate signature is the following (ignoring the WTS on $m_w$ for simplicity):

\begin{equation*}
    \mathcal{R}_{AS} = 
\Bigg\{\space \begin{matrix}
         x = (c_{\mathbf{m}}, t,  c_{\mathbf{pk}, c_{\mathbf{w}}}, c_{\mathbf{S}}, c_{\mathbf{\sigma}_{\mathbf{S}}})\\
         w = (\mathbf{m}, \mathbf{pk}, \mathbf{w}, \mathbf{S}, \mathbf{\sigma_S})\\
    \end{matrix}\quad \Bigg| \quad \begin{matrix}
        \mathbf{pk}\in \mathbb{G}_2^n; c_{\mathbf{pk}} = \mathsf{commit}(\mathbf{pk})\\
        \mathbf{w}\in \mathbb{F}^n, ||\mathbf{w}||_1<|\mathbb{F}|;  c_{\mathbf{w}} = \mathsf{commit}(\mathbf{w})\\
        \forall i\in \{1,2,\dots, k\}, S_i\subseteq \{1,2,\dots, n\}; c_{S} = \mathsf{commit}(\mathbf{S})\\
        \forall i\neq j, S_i \cap S_j = \emptyset \\
        \forall i\in \{1,2,\dots, k\}, \mathbf{\sigma}_{S_i} \in \mathbb{G}_1^{|S_i|} ;c_{\mathbf{\sigma_{S}}} = \mathsf{commit}(\mathbf{\sigma_S}) \\
        \forall i\in \{1,2,\dots, k\}, \forall j\in S_i, e(H(m_i), \mathsf{pk}_j) = e(\sigma_j, g_2) \\
        t = \sum_{i=1}^k \sum_{j\in S_i} w_j  \\
    \end{matrix}\space \Bigg\}
\end{equation*}

\paragraph{Special SNARK} Besides implementing the aggregate signature scheme using generic SNARKs, it is meaningful to characterize its concrete performance and explore special SNARKs that possibly improve the efficiency further. 

\section{Topic 2}
\section{Future Work: Incremental DKG} \label{sec:future-dkg}

Inspired by the application of blockchains, where the network participants are dynamic in the sense that (1) new participants can join and existing participants might not be active, (2) the weights of participants might evolve over time as their stake changes. In case we run key generation protocols very frequently because of the aforementioned dynamic updates, it is meaningful to design suitable DKG protocol that has good amortized complexity that outperforms the trivial approach to rerun the complete DKG protocol again. 

\paragraph{For SNARK-based threshold signature} Threshold signature schemes based on SNARKs do not maintain secret shares of keys, so that DKG is not required for setting up the keys. However, special SNARKs like \cite{DBLP:conf/ccs/DasCXNB023} require the aggregator to interact with signers and do precomputation to improve the efficiency per signature. It is interesting to reduce the amount of precomputation when the set of signers changes. 

\subsection{For secret-sharing based threshold signatures} 

\paragraph{Adding a member} There is a straightforward solution if we want to add a new member $P_{n+1}$ to the group, while keeping the threshold $t$ unchanged. This is because the secret key share of $P_{n+1}$ is nothing but $f(n+1)$, where $f$ is the polynomial such that $f(0)$ is the secret key and $f(i)$ is the secret key share of $P_i$ for $i\in\{1,\dots, n\}$. At least $t$ existing members are required to send to $P_{n+1}$ their reconstruction shares for $f(n+1)$. The secret shares of existing members remain the same. 

\paragraph{Removing a member} 
Removing an existing member is more complicated than adding a new member. If we want to keep the same secret key $f(0)=x$ and threshold $t$, the shares of remaining members must change. Suppose the member to remove is $P_n$, the remaining members should replace $f$ with another function $f'$ such that $f'(0)=f(0)=x$ but $f'(n)$ is indistinguishable from uniform randomness in the view of $P_n$. Since $f'$ and $f$ are different univariate polynomials of degree at most $t-1$, the valuation of $f'$ and $f$ coincide with at most $t-1$ locations, indicating that at least $(n-1)-(t-2) = n-t+1 = \Theta(n)$ remaining members must update their secret key shares. If there is no specific optimization, this is close to rerun the DKG protocol again, such as in \cite{DBLP:journals/npa/NoackS09}. 

Considering these challenges, it is interesting to design novel threshold signature schemes and DKG protocols that have desirable amortized complexity in a dynamic system where members might join or leave. 



\section{Conclusion}



% \begin{theorem}
% This is a sample theorem. The run-in heading is set in bold, while
% the following text appears in italics. Definitions, lemmas,
% propositions, and corollaries are styled the same way.
% \end{theorem}

\newpage

\bibliographystyle{splncs04}
\bibliography{reference}

\end{document}
