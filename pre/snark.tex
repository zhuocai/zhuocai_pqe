\section{SNARKs}
%\frame{\tableofcontents[currentsection]}

\begin{frame}{Limitations of BLS threshold signature}
\begin{itemize}
    \item Unweighted: every member has the same unit weight. \\
        - In cryptocurrencies (Proof-of-Stake), accounts have (vastly) different weights. \\
        - Virtualization approach: suppose $P_1$ has weight 1 and $P_2$ has weight $10,000$, then $P_2$ should own $10,000$ secret shares in TS. \pause 
    \item Fixed threshold: each polynomial (secret shares) corresponds to a fixed threshold $t$. For another threshold, we should setup another polynomial (and secret shares). 
\end{itemize}
\end{frame}



\begin{frame}{Succinct Non-interactive ARgument of Knowledge (Informal)}
    \textbf{Verifiable computing}: for some computation $f$, while computing/evaluating $f(x)$ given input $x$ might take $1$ year, given $(x, y)$ verifying that $y=f(x)$ might take only $1$ second. 
    
    SNARKs approach: firstly compile the computation to a circuit satisfiability problem (CSP). 
    
    \begin{block}{Circuit Satisfiability Problem}
    Arithmetic circuit $C$. \\
    Language $\mathcal{L}_{C}$: $\{ x: \exists$ a witness $w$, such that $C(x, w) = 0\}$.  \\
    Relation $\mathcal{R}_C$: $\{(x, w): C(x, w)=0\}$. 
    \end{block}
    
    $\mathcal{P}$ proves to $\mathcal{V}$ that $x\in \mathcal{L}_C$. \\
    
        
    \end{frame}


    \begin{frame}{SNARKs (continued)}
        \textbf{Polynomial commitment scheme}: the witness $w$ might be long. $\mathcal{P}$ represents $w$ using a polynomial extension $\tilde{w}$, and only sends an $O(1)$ sized commitment of $\tilde{w}$ to $\mathcal{V}$. When $\mathcal{V}$ queries $\tilde{w}(r)$, $\mathcal{P}$ replies with the value and a proof, both succinct. 
        
        \textbf{Interactive Oracle Proofs} (for CSP): $\mathcal{P}$ and $\mathcal{V}$ interact a few round, $\mathcal{V}$ queries the polynomial committed by $\mathcal{P}$. While $\mathcal{P}$ might check $|C|$ constraints, $\mathcal{V}$ only checks a few ($O(1)$). 
        
        \textbf{Non-interactive} a proof can be reused for many different verifiers, that do not need to interact with the prover. 
        
        \begin{itemize}
            \item Proof size: a succinct commitment $c_w$ of $w$ + a SNARK proof $\pi$. $|c_w|$ and $\pi$ can be $O(1)$ field elements. 
            \item Verification time: $O(1)$ field operation + $|x|$ to read the input.
            \item Prover time: ideally only slightly more than the time to evaluate the circuit.  
        \end{itemize}
        \end{frame}


    

\begin{frame}{Generic SNARK for weighted, multi-threshold TS}
    $\mathbf{pk} = [\mathsf{pk}_1, \mathsf{pk}_2,\dots, \mathsf{pk}_n]$, $\mathbf{w} = [w_1, w_2, \dots, w_n]$ \\
    The verifier agree and can access $\mathbf{pk}$, $\mathbf{w}$ via succinct commitments. \\ \pause
    \vspace{0.5em}
    A subset $S$ of $\{P_1, \dots, P_n\}$ generate partial signatures $\sigma_i = H(m)^{x_i}$. \\ \pause
    
    \begin{block}{CSP for WTS}
    Input for $\mathcal{V}$ to read: $x = (m, t, c_{\mathbf{pk}}, c_{\mathbf{w}}, c_{S}, c_{\sigma_{S}})$ \\
    The witness: $w = (\mathbf{pk}, \mathbf{w}, S, \sigma_S)$ \\
    The relation $\mathcal{R}_{WTS}$: 
    \begin{equation*}
        \begin{matrix}
            \mathbf{pk}\in \mathbb{G}^n; c_{\mathbf{pk}} = \mathsf{commit}(\mathbf{pk})\\
            \mathbf{w}\in \mathbb{F}^n, ||\mathbf{w}||_1<|\mathbb{F}|;  c_{\mathbf{w}} = \mathsf{commit}(\mathbf{w})\\
            S\subseteq \{1,2,\dots, n\}; c_{S} = \mathsf{commit}(S)\\
            \sigma_{S} \in \mathbb{G}^{|S|} ;c_{\sigma_{S}} = \mathsf{commit}(\sigma_S) \\
            \forall i\in S, e(H(m), \mathsf{pk}_i) = e(\sigma_i, g) \\
            t = \sum_{i\in S} w_i  \\
        \end{matrix}
    \end{equation*}
    \end{block}
    
    \end{frame}




\begin{frame}{Specialized SNARK for weighted, multi-threshold TS~\cite{DBLP:conf/ccs/DasCXNB023}\footfullcite{DBLP:conf/ccs/DasCXNB023}}
    \textbf{Special SNARKs} Generic SNARKs are designed for a general class of problems (e.g., arithmetic circuit satisfiability problem). They might not be optimal for a particular class of problems. \\
    % \pause 
    % [Das etc. CCS'23] designs a weighted, multi-threshold TS using inner product argument. 
    
    \textbf{Inner Product Argument (IPA)}: prove that $\langle \mathbf{a}, \mathbf{b}\rangle =c$, $\mathbf{a},\mathbf{b}\in \mathbb{F}^n$. Verifiers can access the commitment of $\mathbf{a}$ and $\mathbf{b}$. 
    \pause
    \begin{itemize}
        \item  Let $\mathbf{b} = [b_1, b_2,\dots, b_n]\in \{0,1\}^n$, $b_i=1$ if $P_i$ generates a partial signature $\sigma_i = H(m)^{x_i}$. 
        \item The aggregate signature $\sigma_{\mathbf{b}} = \prod_{b_i=1} \sigma_i = H(m)^{\langle\mathbf{x}, \mathbf{b}\rangle}$. Verification key is $\prod_{b_i=1}y_i = g^{\langle\mathbf{x}, \mathbf{b}\rangle}$. \\
            - However, the aggregator should not learn the secret keys $x_i$. So they use general IPA for $\langle\mathbf{\sigma}, \mathbf{b}\rangle$ and $\langle\mathbf{y}, \mathbf{b}\rangle$. One vector is in the field $\mathbb{F}$, the other in the group $\mathbb{G}$. 
        \item The total weight is $\langle\mathbf{w}, \mathbf{b}\rangle$. 
    \end{itemize}
    
    
    Result : the prover complexity is more practical. 
    \end{frame}